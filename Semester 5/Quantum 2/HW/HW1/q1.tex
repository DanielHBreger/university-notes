\section{Question 1}

\subsection{Part 1}

\begin{colbox}
		Show that
		\begin{equation*}
			\left\lvert\int\limits_0^t \dd t' e^{iwt'}\right\rvert^2 = \frac{\sin\left(\frac{wt}{2}\right)}{\left(\frac{\omega}{2}\right)^2}
		\end{equation*}
\end{colbox}

\begin{align}
	\int\limits_0^t\dd t' e^{i\omega t'} &= \left.\left(\frac{1}{i\omega}e^{i\omega t'}\right)\right\rvert^{t'=t}_{t'=0}\\
	&= \frac{1}{i\omega}\left(e^{i\omega t}-1\right)\\
	&\therefore \\
	\left\lvert\int\limits_0^t \dd t' e^{iwt'}\right\rvert^2 &= \left\lvert\frac{1}{i\omega}\left(e^{i\omega t}-1\right)\right\rvert^2\\
	&= \frac{1}{i\omega}\left(e^{i\omega t}-1\right) \cdot \overline{\frac{1}{i\omega}\left(e^{i\omega t}-1\right)}\\
	&= \frac{1}{i\omega}\left(e^{i\omega t}-1\right) \cdot \frac{-1}{i\omega}\left(e^{-i\omega t}-1\right)\\
	&= \frac{1}{\omega^2}\left(2-e^{i\omega t}-e^{-i\omega t}\right)\\
	&= \frac{1}{\omega^2}\ins{2-2\cos(\omega t)}\\
	&= \frac{4\sin^2\ins{\frac{\omega t}{2}}}{\omega^2}\\
	&= \frac{\sin^2\ins{\frac{\omega t}{2}}}{\ins{\frac{\omega}{2}}^2}
\end{align}

\clearpage
\subsection{Part 2}
\begin{colbox}
	Calculate the limit
	\begin{equation*}
		\lim\limits_{t\rightarrow\infty}\frac{\sin^2(\alpha t)}{\alpha^2 t}
	\end{equation*}	
	for both $\alpha = 0$ and $\alpha \neq 0$. Conclude that
	\begin{equation*}
		\lim\limits_{t\rightarrow\infty}\frac{\sin^2(\alpha t)}{\alpha^2 t} \propto \delta(\alpha)
	\end{equation*}
	and find the proportional constant using
	\begin{equation*}
		\frac{1}{\pi}\int_{-\infty}^\infty \dd\xi \frac{\sin^2\xi}{\xi^2}=1
	\end{equation*}
\end{colbox}

\subsubsection{$\alpha=0$}

Since the limit when alpha is zero isn't well defined, we'll take the limit of alpha approaching zero first.

\begin{align}
	\left.\lim\limits_{t\rightarrow\infty}\frac{\sin^2(\alpha t)}{\alpha^2 t} \right\rvert_{\alpha=0} &= \lim\limits_{\alpha\rightarrow0}\lim\limits_{t\rightarrow\infty}\frac{\sin^2(\alpha t)}{\alpha^2 t} \label{sined}\\
	&= \lim\limits_{t\rightarrow\infty}\frac{(\alpha t)^2}{\alpha^2 t} \label{unsined} \\
	&= \lim\limits_{t\rightarrow\infty} t\\
	&= \infty
\end{align}

The transition between \ref{sined} and \ref{unsined} being because the sine is very small.

\subsubsection{$\alpha \neq 0$}

\begin{align}
	0=\lim\limits_{t\rightarrow\infty}-\frac{1}{\alpha^2 t}\leq\lim\limits_{t\rightarrow\infty}\frac{\sin^2(\alpha t)}{\alpha^2 t} \leq \lim\limits_{t\rightarrow\infty}\frac{1}{\alpha^2 t} = 0
\end{align}

Therefore

\begin{equation}
	\lim\limits_{t\rightarrow\infty}\frac{\sin^2(\alpha t)}{\alpha^2 t} = \begin{cases}
		\infty & \alpha=0\\
		0 & \alpha \neq 0
	\end{cases}
\end{equation}

which is the exact behavior of the delta function. To find the proportion constant we solve

\begin{align}
	\int_{-\infty}^\infty \frac{\sin^2(\alpha t)}{\alpha^2 t} \dd \alpha = [x=\alpha t] &= \int_{-\infty}^\infty \frac{\sin^2(x)}{x^2} \dd x\\
	&= \pi
\end{align}

therefore

\begin{equation}
	\lim\limits_{t\rightarrow\infty}\frac{\sin^2(\alpha t)}{\alpha^2 t} = \pi\delta(\alpha) \label{1.2}
\end{equation}

\subsection{Part 3}
\begin{colbox}
	Use the previous parts to show that
	\begin{equation*}
		\lim\limits_{t\rightarrow\infty}\frac{\sin^2\ins{\frac{\omega t}{2}}}{\ins{\frac{\omega}{2}}^2} \sim \pi t \delta\ins{\frac{\omega}{2}}
	\end{equation*}
\end{colbox}

if we insert $\alpha = \frac{\omega}{2}$ into \ref{1.2} we get the correct expression.

\subsection{Part 4}
\begin{colbox}
	Calculate the limit
	\begin{equation*}
		\lim\limits_{\lambda\rightarrow0}\frac{\lambda}{\alpha^2+\lambda^2}
	\end{equation*}
	for $\alpha=0$ and $\alpha\neq 0$ and conclude that
	\begin{equation*}
		\lim\limits_{\lambda\rightarrow0}\frac{\lambda}{\alpha^2+\lambda^2} \propto \delta(\alpha)
	\end{equation*}
	and find the proportion constant using
	\begin{equation*}
		\frac{1}{\pi}\int_{-\infty}^\infty \dd\alpha \frac{\xi}{\alpha^2+\xi^2}=1
	\end{equation*}
\end{colbox}

\begin{align}
	\left.\lim\limits_{\lambda\rightarrow0}\frac{\lambda}{\alpha^2+\lambda^2}\right\rvert_{\alpha=0} &= \lim\limits_{\lambda\rightarrow0}\frac{1}{\lambda} = \infty
\end{align}

\begin{align}
	\lim\limits_{\lambda\rightarrow0}\frac{\lambda}{\alpha^2+\lambda^2} = [\alpha\neq0] = 0
\end{align}

Therefore

\begin{equation}
	\lim\limits_{\lambda\rightarrow0}\frac{\lambda}{\alpha^2+\lambda^2} \propto \delta(\alpha)
\end{equation}

and to find the proportion constant

\begin{align}
	\int_{-\infty}^\infty \frac{\lambda}{\alpha^2+\lambda^2} &= \pi
\end{align}

therefore

\begin{equation}
	\lim\limits_{\lambda\rightarrow0}\frac{\lambda}{\alpha^2+\lambda^2} = \pi\delta(\alpha)
\end{equation}