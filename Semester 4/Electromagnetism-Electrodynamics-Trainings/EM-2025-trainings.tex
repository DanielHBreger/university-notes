\documentclass[english,course]{lecturenotes}
\usepackage{amsmath}
\usepackage{amsfonts}
\usepackage{amssymb}
\usepackage{tensor}
\usepackage{mathtools}
\usepackage{witharrows}
\usepackage{diffcoeff}
\usepackage{cancel}
\usepackage[integrals]{wasysym}
%
% First, provide some data about this document
\title{Electromagnetism \& Electrodynamics}
\subtitle{Trainings}
\shorttitle{E\&M Trainings} % For headers; if undefined, the usual title will be used
\ccode{01140246} % Most of these data are not compulsory
\subject{Subject of the Talk}
\author{Daniel Haim Breger}
\spemail{shai.kapon@campus.technion.ac.il}
\email{danielbreger@campus.technion.ac.il}
\speaker{Shai Kapon}
\date{1}{04}{2025}
\dateend{13}{07}{2025}
\conference{Ullmann 309}
\place{Technion}
% \flag{Sample \texttt{lecturenotes.cls} v3}
\attn{These notes were typed during the 2025 spring semester in the Technion. The lectures were given in Hebrew and were live-translated by me to English. The document is provided as is and likely contains many errors. This document contains session 4-9,12-13. The first three were missed and session 10-11 happened over zoom due to the war with Iran, official notes should be available for them on the drive.}
% \morelink{vhbelvadi.com}
%
% And then begin your document

\begin{document}
\newcommand{\tsr}[2]{\tensor{#1}{#2}}
\newcommand{\mtr}[1]{\begin{pmatrix}\end{pmatrix}}
\newcommand\dd{\mathop{}\!\mathrm{d}}
\newcommand{\lagr}{\mathcal{L}}
\newcommand{\ins}[1]{\left(#1\right)}
\newcommand{\holdpage}[0]{d\\d\\d\\d\\d\\d\\d\\d\\d\\d\\d\\d\\d\\d\\d\\d\\d\\d\\d\\d\\d}
\clearpage

\section{Classical field theory}
\lecture[2 hours]{29}{04}{2025}

The action of the EM field is given by


\begin{equation*}
    S=S_f+S_{int}+S_A= -mc\int d\tau +\frac{e}{c}\int dx^\mu A_\mu + S_A
\end{equation*}

Where

\begin{align*}
    \tensor{F}{_\mu_\nu} &= \partial_\mu A_\nu - \partial_\nu A_\mu\\
    \tensor{\Tilde{F}}{^\rho^\sigma}&=\frac{1}{2}\frac{\tensor{\epsilon}{^\rho^\sigma^\mu^\nu}}{|\sqrt{g}|}\tensor{F}{_\mu_\nu}\\
    \partial_\rho F_{\mu \nu}+\partial_\nu F_{mu}+\partial_\mu F_{\nu \rho} &= 0\\
    S_A &= -\frac{1}{16\pi c}\int d^4x \tensor{F}{_\mu_\nu}\tensor{F}{^\mu^\nu}
\end{align*}

\subsection{Variation on the fields}

\begin{equation*}
    \frac{\delta \Phi_i(x^\mu)}{\delta\Phi_j(x'^\mu)} = \delta^{D+1}(x^\mu - x'^\mu)\tensor{\delta}{^j_i}
\end{equation*}

To do the variation we do the usual variation but requiring that the edges of the integration are that $\delta A^\mu=0$ both at the initial times and at every time when $|x|\rightarrow\infty$. Therefore

\begin{align*}
    S &= \int dtL= \int d^4 x\lagr\\
    &= \int dx^0dx^1dx^2dx^3\lagr = \int cdt d^3x\lagr
\end{align*}

We'll look at a system of masses on springs and take that to the continuum limit. The masses are of mass m, the spring has spring constant k, and the offset from equilibirium for each mass is $\eta_i$

First we'll write down the lagrangian of the system

\begin{equation}\label{firsteq}
    \lagr = T-V = \sum\limits_i \frac{m}{2}(\eta_i)^2 - \frac{k}{2}\sum\limits_i(\eta_{i+1}-\eta_i)^2
\end{equation}

Therefore the Euler-Lagrange equation is

\begin{equation*}
    0 = \diff*{}t(\diffp{L}{\dot{\eta_i}})-\diffp{L}{\eta_i}
\end{equation*}

and the resulting equations of motion are

\begin{equation*}
    0 = m\Ddot{\eta_i} - K(\eta_{i+1}-\eta_i)+K(\eta_i-\eta_{i-1})
\end{equation*}

and at the continuum limit where $\mu = \frac{m}{a}$ and with the Youngs modulus which we define as $\mathcal{Y} = Ka$ and the strain $\xi = \frac{\eta_{i+1}-\eta_i}{a}$ we get

\begin{equation*}
    F = \mathcal{Y}\xi
\end{equation*}

Which is exactly Hooke's law but for continuous solids. and next we can generalize equation \ref{firsteq} to the continuum limit. The kinetic part is thus

\begin{align*}
    \sum\limits_i \frac{m}{2}(\eta_i)^2 &= \sum\limits_i \frac{m}{a}a(\eta_i)^2\\
    &= \frac{1}{2}\int\mu dx (\partial_t\eta(x,t))^2
\end{align*}

and the potential part is

\begin{align*}
    \frac{k}{2}\sum\limits_i(\eta_{i+1}-\eta_i)^2 &= \frac{1}{2}\sum aka(\eta_{i+1}-\eta_i)^2\\
    &= \frac{1}{2}\sum\mathcal{Y}a(\eta_{i+1}-\eta_i)^2\\
    &= \frac{1}{2}\int\mathcal{Y}dx(\partial_x\eta(x,t))^2
\end{align*}

Therefore the Lagrangian density is

\begin{equation}
    \lagr = \frac{\mu}{2}(\partial_t\eta)^2-\frac{\mathcal{Y}}{2}(\partial_x\eta)^2
\end{equation}

and the action is 

\begin{equation*}
    S = \iint dtdx\lagr
\end{equation*}

and we could find the Lagrangian by

\begin{equation*}
    L=\int dx\lagr
\end{equation*}

for obvious reasons we expect the resulting equations of motion to be the wave equation, which we'll show in the homework.

\subsection{The Euler-Lagrange equations of the fields}

Say we have the action of a scalar field

\begin{equation}
    S[\Phi] = \int \lagr(\Phi,\partial_\mu\Phi)d^4x
\end{equation}

and thus the EL equations are

\begin{equation}
    \diffp{L}\Phi - \partial_\mu\left(\diffp{\lagr}{\partial_\mu\Phi}\right) = 0
\end{equation}

We can develop this from the variation principle

\begin{align*}
    0 = \delta S &= \int d^4x (-\lagr(\Phi,\partial_\mu\Phi)+\lagr(\Phi+\delta\Phi,\partial_\mu(\Phi+\delta\Phi)))
\end{align*}

And we can approximate

\begin{equation*}
    \lagr(\Phi+\delta\Phi,\partial_\mu(\Phi+\delta\Phi)) \approx\lagr(\Phi,\partial_\mu\Phi)+\diffp{\lagr}{\Phi}\delta\Phi+\diffp{\lagr}{\partial_\mu\Phi}\delta\partial_\mu\Phi
\end{equation*}

therefore

\begin{align*}
    \delta S &= \int d^4x\delta\lagr = \int d^4x \left(\lagr(\Phi,\partial_\mu\Phi)+\diffp{\lagr}{\Phi}\delta\Phi+\diffp{\lagr}{\partial_\mu\Phi}\partial_\mu\delta\Phi\right)\\
    &= \int d^4x\left(\diffp{\lagr}{\Phi}-\partial_\mu\diffp{\lagr}{\partial_\mu\Phi}\right)\delta\Phi = 0
\end{align*}

Therefore

\begin{equation*}
    \diffp{\lagr}{\Phi}-\partial_\mu\diffp{\lagr}{\partial_\mu\Phi} = 0
\end{equation*}

We can generalize the el equations to an action of multiple fields

\begin{equation*}
    \diffp{\lagr}{\Phi} - \partial_\mu\left(\diffp{\lagr}{\partial_\mu\Phi}\right) = 0
\end{equation*}


\subsection{The Momentum-Energy Tensor}

This will be brief. The momentum-energy tensor is defined by

\begin{equation*}
    \tensor{T}{^\nu_\mu} = \diffp{L}{\partial_\nu\varphi_i}\diffp{\varphi_i}{x^\mu}-\lagr\tensor{\delta}{^\nu_\mu}
\end{equation*}

And when the Lagrangian is not explicitly dependent on the coordinates

\begin{equation*}
    \partial_\eta \tensor{T}{^\nu_\mu} = 0
\end{equation*}

The tensor's elements are

\begin{equation*}
    P^i = \frac{1}{c}\int d^3x\tensor{T}{^i^0}
\end{equation*}

and

\begin{equation*}
    W = \tensor{T}{^0^0}
\end{equation*}

which is the energy density.

\subsection{Noether's Theorem}

We'll assume we have an action $S[\Phi]$ and that we have some variation upon $\varphi \rightarrow\varphi+\epsilon\psi$ which changes the Lagrangian such that

\begin{equation*}
    \lagr \rightarrow \lagr + \epsilon\partial_\mu K^\mu
\end{equation*}

Which gives us a conserved current

\begin{align*}
    j^\nu = K^\nu-\diffp{\lagr}{\partial_\eta\varphi}\psi && \partial_\nu j^\nu = 0
\end{align*}

We'll now assume we have a lagrangian $\lagr(\Phi, \partial_\mu\Phi)$ that's invariant to a certain variation $\Phi \rightarrow \Phi + c_0$. This means that $\lagr(\partial_\mu\Phi)$. We will want to show that there is a conservation law. We'll notice that we can define

\begin{align*}
    \epsilon = c_0 && \psi = 1 && K^2=0
\end{align*}

Therefore our conserved current is

\begin{equation*}
    j^\nu = 0-\diffp{\lagr}{\partial_\nu\Phi}\cdot1 = -\diffp{\lagr}{\partial_\nu\Phi}
\end{equation*}

Now we'll assume the existence of a lagrangian that's invariant for variations by a constant, i.e

\begin{equation*}
    \lagr(\Phi+c_0) = \lagr(\Phi)
\end{equation*}

We'll perform the following variation:

\begin{align*}
    \Phi \rightarrow \Phi+\delta \Phi(x)
\end{align*}

Which is not a symmetry of the system, but in the limit where it's constant is. Then we'll perform variation on the action, and conclude the conservation law this grants us.

We'll begin by writing down the variation.

\begin{align*}
    \delta S &= \int \delta \lagr d^4x = \int d^4x\diffp{\lagr}{\partial_\mu \Phi}\delta\partial_\mu \Phi\\
    &= \int d^4x\diffp{\lagr}{\partial_\mu \Phi}\partial_\mu\delta \Phi\\
    = \text{I.B.P.} &= -\int \partial_\mu \diffp{\lagr}{\partial_\mu \Phi}\delta \Phi d^4x\\
    &= -\int \partial_\mu J^\mu \delta\Phi d^4x = 0
\end{align*}

Therefore 

\begin{equation*}
    j^\mu = \diffp{\lagr}{\partial_\mu\Phi}
\end{equation*}

\subsection{Example}

The following action is given

\begin{equation*}
    S[\Phi] = \int \left(\frac{1}{2}\partial^\mu\varphi\partial_\mu\varphi-x\varphi^4\right)d^4x
\end{equation*}

\subsubsection{part 1}
The following transformation is given, check if it's a symmetry of the action
\begin{align*}
    \varphi \rightarrow \varphi + x^\mu\partial_\mu\varphi
\end{align*}

We'll define $\varphi' = \varphi + x^\mu\partial_\mu\varphi$ and thus

\begin{align*}
    \partial_\nu\varphi' &= \partial_\nu\varphi + \partial_\nu(x^\mu\partial_\mu\varphi)\\
    &= \partial_\nu\varphi + (\partial_\nu x^\mu)\partial_\mu \varphi + x^\mu\partial_\nu\partial_\mu\varphi\\
    &= \partial_\nu\varphi + \tensor{\delta}{^\mu_\nu}\partial_\mu\varphi+x^\mu\partial_\nu\partial_\mu\varphi\\
    &= 2\partial_\nu\varphi + x^\mu\partial_\nu\partial_\mu\varphi
\end{align*}

Next we'll look at the kinetic part

\begin{align*}
    \frac{1}{2} \int d^4x (\partial^\mu\varphi'\partial_\mu\varphi') &= \frac{1}{2}\int d^4x (2\partial^\mu\varphi + x^\nu\partial^\mu\partial_\nu\varphi)(2\partial_\mu\varphi + x^\sigma\partial_\mu\partial_\sigma\varphi)
\end{align*}

And after some math we get elements of the form $\partial_\mu(x^\nu\partial_\nu\varphi)$ then with some more development we get that $\lambda\varphi^4$ will be symmetries.


\subsection{The Momentum-Energy Tensor (again)}
\lecture[2 hours]{6}{05}{2025}

First we'll write Maxwell's equations

\begin{align}
	\nabla\cdot B &= 0 && \nabla\cdot E = 4\pi\rho\\
	\nabla\times E &=-\frac{1}{c} \diffp Bt && \nabla\times B = \frac{1}{c} \diffp Et +\frac{4\pi}{c}J
\end{align}

and in index form

\begin{align}
	\partial_\rho \tsr{\tilde F}{^\rho^\sigma} = 0 && \partial_\mu \tsr {F}{^\mu^\nu} = \frac{4\pi}{c}J^\nu\\
	\partial_\rho \tsr{\epsilon}{^\sigma^\rho^\mu^\nu}\tsr{F}{_\mu_\nu} = 0
\end{align}

and the following is called Poynting's law

\begin{equation}
  \frac{d}{dt}\left(\frac{E^2+B^2}{8\pi}\right) = -J\cdot E -\nabla\cdot S
\end{equation}

We'll integrate over the entirety of space

\begin{equation}
  \int J\cdot E dV=\sum ev\cdot E = \sum \diff* {\epsilon_{kinetic}} t
\end{equation}

and thus 

\begin{equation}
  \frac{d}{dt} \left[\int\frac{E^2+B^2}{8\pi}dV + \sum {\epsilon_{kinetic}} \right] = 0
\end{equation}

and using Gauss' law

\begin{equation}
  -\int \nabla\cdot SdV = \oiint SdA
\end{equation}

The energy density of the electromagnetic field is

\begin{equation}\label{energydensity}
  W=\frac{E^2+B^2}{8\pi}
\end{equation}

We'll integrate over a finite part of space

\begin{equation}
  \frac{d}{dt} \left[\int\frac{E^2+B^2}{8\pi}dV + \sum {\epsilon_{kinetic}} \right] = - \oiint SdA
\end{equation}

The momentum energy tensor is

\begin{equation}
  \tsr{T}{_\mu^\nu} = \sum q_\mu^{(i)} \diffp{\lagr}{q_\nu^{(i)}} - \tsr{\delta}{_\mu^\nu}\lagr
\end{equation}

And if there are no sources then

\begin{equation}\label{nosources}
  \partial_\nu \tsr{T}{_\mu^\nu}=0
\end{equation}

We can also split this into two equations

\begin{align}\label{firstofpairconserv}
	\frac{1}{c}\diffp{\tsr{T}{^0^0}}{t}+\diffp{\tsr{T}{^0^i}}{{x^i}} = 0
\end{align}

and

\begin{equation}\label{secondofpairconserv}
  \frac{1}{c}\diffp{\tsr{T}{^i^0}}{t}+\diffp{\tsr{T}{^i^j}}{{x^j}} = 0
\end{equation}

We'll first focus on equation \ref{firstofpairconserv} and integrate it using Gauss' law

\begin{equation}
  \frac{d}{dt}\int \tsr{T}{^0^0}dV = -c\oiint \tsr{T}{^o^i}da_i
\end{equation}

We can call the left integrand the energy density and the right integrand the energy current density. This resolves to equation \ref{energydensity} when 

\begin{equation}
  \lagr = \lagr_A=-\frac{1}{16\pi c}\tsr{F}{_\mu_\nu}\tsr{F}{^\mu^\nu}
\end{equation}

Therefore $\tsr{T}{^i^0}$ is the Poynting vector, and T looks thusly:

\begin{equation}
T=\begin{pmatrix}
	W&\frac{S_x}{c}&\frac{S_y}{c}&\frac{S_z}{c}\\\frac{S_x}{c}&&&\\\frac{S_y}{c}&&&\\\frac{S_z}{c}&&&
\end{pmatrix}
\end{equation}

We'll now look at integrating equation \ref{secondofpairconserv}:

\begin{equation}
  \frac{d}{dt} \int \frac{1}{c}\tsr{T}{^u^0}dV = \oiint \tsr{T}{^i^j}da_j
\end{equation}

where the right integrand is the momentum current density.

\begin{figure*}[!htb]
	\centering
	\includegraphics[width=0.75\textwidth]{Stress-tensor-t-ij-in-a-3D-space.ppm}
	\caption{The Stress-Energy Tensor}
	\label{stress-energy-tensor-cube}
\end{figure*}

Therefore we now know that the stress energy tensor looks like

\begin{equation}
  \tsr{T}{^\mu^\nu} = \begin{pmatrix}
	W&\frac{S_x}{c}&\frac{S_y}{c}&\frac{S_z}{c}\\\frac{S_x}{c}&\sigma_{xx}&\sigma_{xy}&\sigma_{xz}\\\frac{S_y}{c}&\sigma_{yx}&\sigma_{yy}&\sigma_{yz}\\\frac{S_z}{c}&\sigma_{zx}&\sigma_{zy}&\sigma_{zz}
\end{pmatrix}
\end{equation}

\subsection{The Stress-Energy Tensor}

Now we'll look at a specific action.

\begin{equation}
  \lagr_A=-\frac{1}{16\pi c} \tsr{F}{_\mu_\nu}\tsr{F}{^\mu^\nu}
\end{equation}

and

\begin{equation}
  \tsr{T}{^\mu^\nu} = \frac{1}{4\pi c}\left[-\tsr{F}{^\mu^\rho}\tsr{F}{_\rho^\nu}+\frac{1}{4}\tsr{\eta}{^\mu^\nu}\tsr{F}{_\rho_\alpha}\tsr{F}{^\rho^\alpha}\right]
\end{equation}

\begin{equation}
  \tsr{\sigma}{_i_j}=\frac{1}{4\pi}\left[ E_iE_j+B_iB_j-\frac{1}{2}\delta_{ij}
(E^2+B^2)\right]
\end{equation}

and the definition of Maxwell's Stress Energy Tensor is $[stress]=\frac{[F]}{[A]}$.

Now we'll look at the situation in equation \ref{nosources} but with sources:

\begin{equation}
  \partial_\nu \tsr{T}{_\mu^\nu} = -\tsr{F}{_\mu_\nu}J^\nu
\end{equation}

Thus Poynting's law when $\mu=0$ is:

\begin{equation}
  \frac{d}{dt}\left(\frac{E^2+B^2}{8\pi}\right) = -J\cdot E -\nabla\cdot S
\end{equation}

and when $\mu=i$:

\begin{equation}
  \partial_0\tsr{T}{_i^0}+\left(\nabla\cdot\overleftrightarrow\sigma\right)_i = \rho E_i+\frac{1}{c}(J\times B)
\end{equation}

Looking at the following alternative writings

\begin{align}
	\partial_0\tsr{T}{_i^0} = \diff{P_{em}}{t} && \rho E_i +\frac{1}{c}(J\times B) = \diff{P_{mech}}{t}
\end{align}

We can rewrite this in the following way

\begin{equation}
  \diff{P_{em}}{t} + \diff{P_{mech}}{t} = \nabla\cdot \overleftrightarrow\sigma
\end{equation}

At this point we were informed there could be minus symbols in various places and to take this all with a grain of salt. We can now integrate and use Gauss' law and find that

\begin{equation}
  \frac{d}{dt}\int P_{tot}dV = \oiint\overleftrightarrow\sigma d^2x
\end{equation}

and now an exercise:

\subsubsection{An exercise}

We are given a sphere of radius R with even surface charge density $\Sigma$. The following electric field is present:

\begin{equation}
  E=\begin{cases}
  	\frac{4\pi\Sigma}{r^2}\hat r & r \geq R\\
  	0 & r < R
  \end{cases}
\end{equation}

Using the stress energy tensor, find the force being applied between the top half of the sphere and its other half.

First the EM momentum is

\begin{equation}
  P_{em}=\int dV \frac{c}{4\pi} E\times B=0
\end{equation}

since there is no magnetic field. Next we know that

\begin{equation}
  \diff{P_{mech}}{t} = \oiint\limits_{\partial V}\overleftrightarrow\sigma d^2x
\end{equation}

We don't have to solve the entire thing, we just need to look at $F_z$, because of symmetry. Therefore we only need 

\begin{equation}
  F_z = \diff{P_{mech,z}}{t} = \oiint\limits_{\partial V}\sigma_{iz}n^id^2x
\end{equation}

The surface we're integrating over is just the top half of the sphere, which means it's constructed of the top half of the sphere's surface as well as a disk going through the middle of the sphere. The integral over the inner disk will be equal to 0, since the field is always in the radial direction, but the disk's normal direction is always in the z direction, thus always perpendicular. Thus we're left with just the integral over the surface. First we'll write down 

\begin{equation}
  \sigma_{ij} = \frac{1}{4\pi}\left[E_iE_j-\frac{1}{2}\delta_{ij}E^2\right]
\end{equation}

\begin{equation}
  n^i=\begin{pmatrix}
  	\sin\theta\cos\varphi\\\sin\theta\sin\varphi\\\cos\theta
  \end{pmatrix}
\end{equation}

\begin{equation}
  \sigma_{xz} = \frac{1}{4\pi}E_xE_z=2\pi\Sigma^2\sin{2\theta}\cos{\varphi}
\end{equation}

\begin{equation}
  \sigma_{yz} = \frac{1}{4\pi}E_yE_z=2\pi\Sigma^2\sin{2\theta}\sin{\varphi}
\end{equation}

\begin{equation}
  \sigma_{zz} = \frac{1}{4\pi}(E_z^2-\frac{1}{2}E^2)=2\pi\Sigma^2\cos{2\theta}
\end{equation}

A surface element is

\begin{equation}
  d^2x=R^2\sin\theta d\theta d\varphi
\end{equation}

thus

\begin{equation}
  F_z = \oiint\limits_{\partial V}2\pi \Sigma^2\cos\theta R^2\sin\theta d\theta d\varphi = 2(\pi\Sigma R)^2
\end{equation}

\clearpage
\section{Electrostatics}
\subsection{10 minutes on gauge symmetry}

Lorentz gauge condition: 

\begin{equation}
  \partial_\mu A^\mu=0
\end{equation}

Coulomb gauge condition:

\begin{equation}
  \partial_iA^i=0
\end{equation}

temporal gauge

\begin{equation}
  A_0=0
\end{equation}

Axial gauge

\begin{equation}
  A_3=0
\end{equation}

And the equations of motion under Coulomb gauge are

\begin{equation}
  \nabla^2A^0=-4\pi\rho
\end{equation}

\begin{equation}
  \Box A^i-\partial^i\partial_0A^0=\frac{4\pi}{c}J^i
\end{equation}

In electrostatics we deal with systems that have the following conditions:

\begin{align}
	B=0 && J^i=0
\end{align}

i.e there is no magnetic field and there are no currents.

\subsection{A short exercise}

Write the equations of motion in electrostatics in temporal gauge.

\begin{equation}
  \tsr{F}{_i_j}=0 \rightrightarrows \partial_iA_j-\partial_jA_i=0 \rightrightarrows \partial_iA_j=\partial_jA_i
\end{equation}

and

\begin{equation}
  -\frac{4\pi}{c}J^\nu = -\Box A^\nu+\partial^\nu\partial_\mu A^\mu
\end{equation}

index 0 is 0 so we can rewrite

\begin{align}
  -\frac{4\pi}{c}J^i &= -\Box A^i+\partial^i\partial_\mu A^\mu\\
  &= -\Box A^i + \cancel{\partial^i\partial_0A^0} +\partial^i\partial_jA^j\\
  &= -\partial_\nu\partial^\nu A^i+\partial^i\partial_jA^j\\
  &= -\partial_\nu\partial^\nu A^i+\partial_j\partial^iA^j\\
  &= -\partial_0\partial^0A^i-\partial_j\partial^jA^i+\partial_j\partial^iA^i\\
  &= -\partial_0\partial^0 A^i
\end{align}

\lecture[2 hours]{20}{05}{2025}

\section{solving the 2023 midterm}

The following is given: a 4-potential $A_m$ and an indexless field $\Phi$. The actions given are

\begin{align}
	S&=S_A+S_\Phi+S_{A,\Phi}\\
	S_A&=-\frac{1}{16\pi c}\int \dd^4x\tsr{F}{_\mu_\nu}\tsr{F}{^\mu^\nu}\\
	S_\Phi&=\int \dd^4x \left(\frac{1}{2}\partial_\mu\Phi\partial^\mu\Phi-\frac{1}{2}m_\Phi^2\Phi^2\right)\\
	S_{A,\Phi}&=-\frac{1}{4}\int\dd^4xa\Phi\tsr{F}{_\mu_\nu}\tsr{\tilde F}{^\mu^\nu}
\end{align}

\subsection{What are the units of Phi and a in terms of time mass and length?}

\begin{equation}
  [S]=\frac{M\cdot L^2}{T}
\end{equation}

and

\begin{equation}
  [\partial_\mu\Phi\partial^\mu\Phi]=[m_\Phi^2\Phi^2]
\end{equation}

so

\begin{equation}
  [\partial_\mu\Phi]^2=[m_\Phi]^2[\Phi]^2
\end{equation}

therefore

\begin{equation}
  [m_\Phi]=[\partial_\mu\partial^\mu]
\end{equation}

returning to the first one

\begin{equation}
  [S]=\frac{M\cdot L^2}{T} = [S_\Phi]=[\dd^4x\partial_\mu\partial^\mu\Phi\Phi]=L^4L^{-2}[\Phi]^2
\end{equation}

therefore 

\begin{equation}
  [\Phi]=(\frac{M}{T})^{1/2}
\end{equation}

now

\begin{equation}
  [S_{A,\Phi}]=[\dd^4xFFa\Phi]=[\dd^4xFF][a\Phi]=[S_A]\cdot \frac{L}{T}[a\Phi]
\end{equation}

therefore

\begin{equation}
  \frac{T}{L}=[a\Phi]
\end{equation}

and thus

\begin{equation}
  [a] = \frac{T}{L}\frac{1}{[\Phi]}=\frac{T}{L}(\frac{T}{M})^{1/2}
\end{equation}

If you don't know how to do this you can usually figure it out directly, partial derivatives are 1/r  and the potential has its own unit but its usually much easier if you remember the units of the action. This was worth 10 points.

\subsection{Show how $\Phi$ behaves under a Lorentz transformation}

We know that $S_{a,\Phi}$ is a Lorentz scalar because it is an action and in this course we assume all actions are Lorentz scalars, as well as $\dd^4x$, which can be shown to be invariant explicitly but we can just assume it is true. We also know that $F\tilde F$ is a lorentz scalar. Thus if we want to preserve the action under a Lorentz boost it must mean that $\Phi$ is itself a Lorentz scalar. This was also worth 10 points.

\subsection{Mirror transformation}

We define a mirror transformation as follows

\begin{equation}
  (x^0,x^1,x^2,x^3) \rightarrow (x^0,-x^1,-x^2,-x^3)
\end{equation}

use the fact the action is invariant for mirrorings and show how $\Phi$ transforms.

The integrals of our action transform this way:

\begin{equation}
  \iiint\int\limits_{-\infty}^\infty \dd^4x \rightarrow -\iiint\int\limits_{\infty}^{-\infty}=\int\dd^4x(-1)^6
\end{equation}

we also know that \begin{equation}
  F\tilde F \propto \bold E\cdot \bold B \rightarrow -\bold E\cdot \bold B=-F\tilde F
\end{equation}

which means our action transforms as

\begin{equation}
  S_{a,\Phi} \rightarrow \int\dd^4x(-F\tilde F)a\Phi'
\end{equation}

thus we must find that $\Phi'=-\Phi$. This was also 10 points.

\subsection{Find the equations of motion for $\Phi$ and $A_\mu$}

We expect to find either 2 or 5 equations of motions, depending on how you look at it. either one vector equation for A and one for phi, or 4 for A and 1 for phi. Before we solve this we do a small side question: prove and use the following for an antisymmetric C:

\begin{equation}
  \tsr{C}{^\mu^\nu}\delta\tsr{F}{_\mu_\nu}=2\tsr{C}{^\mu^\nu}\partial_\mu\delta A_\nu
\end{equation}

We'll write F explicitly:

\begin{align}
  \tsr{C}{^\mu^\nu}\delta\tsr{F}{_\mu_\nu} &= \tsr{C}{^\mu^\nu}(\partial_\mu\delta A_\nu-\partial_\nu\delta A_\mu)\\
  &= \tsr{C}{^\mu^\nu}\partial_\mu\delta A_\nu - \tsr{C}{^\mu^\nu}\partial_\nu\delta A_\mu\\
  &= \tsr{C}{^\mu^\nu}\partial_\mu\delta A_\nu - \tsr{C}{^\nu^\mu}\partial_\mu\delta A_\nu\\
  &= \tsr{C}{^\mu^\nu}\partial_\mu\delta A_\nu + \tsr{C}{^\mu^\nu}\partial_\mu\delta A_\nu\\
  &= 2\tsr{C}{^\mu^\nu}\partial_\mu\delta A_\nu
\end{align}

where we used changes of index names a few times to get this and that C is antisymmetric.

We'll use variation to find the equations of motion, We prefer EL but Shai promises we'll prefer variation. They ran statistics.

\begin{align}
	\frac{\delta S_A}{\delta \Phi}&=0\\
	\frac{\delta S_{a,\Phi}}{\delta \Phi} &= -\frac{1}{4}\int \dd^4xa\tsr{F}{_\mu_\nu}\tsr{\tilde F}{^\mu^\nu}\delta(x-x')=-\frac{1}{4}a\tsr{\tilde F}{^\mu^\nu}\\
	\frac{\delta S_\Phi}{\delta \Phi}&=\Box\Phi+m_\Phi^2\Phi
\end{align}

so the total variation is

\begin{equation}
  \Box\Phi+m^2_\Phi\Phi=\frac{1}{4}a\tsr{\tilde F}{^\mu^\nu}
\end{equation}

on the other hand

\begin{align}
	\frac{\delta S_A}{\delta A_\rho}&=\frac{1}{4\pi c}\partial_\mu\tsr{F}{^\mu^\nu}\delta_\nu^\rho\\
	\frac{\delta S_\Phi}{\delta A_\rho} &= 0\\
	\frac{\delta S_{A,\Phi}}{\delta A_\rho} &= -\frac{1}{4}\int \dd^4x\Phi\frac{\delta \tsr{\tilde F}{^\mu^\nu}}{\delta A_\rho}\\
	&= -\frac{a}{4}\int \dd^4x\Phi\left[\frac{\delta \tsr{F}{_\mu_\nu}}{\delta A_\rho}\tsr{\tilde F}{^\mu^\nu}+\tsr{F}{_\mu_\nu}\frac{\delta \tsr{\tilde F}{^\mu^\nu}}{\delta A_\rho}\right]\\
	&= -\frac{a}{4}\int \dd^4x\Phi\left[\frac{\delta \tsr{F}{_\mu_\nu}}{\delta A_\rho}\frac{1}{2}\tsr{\epsilon}{^\mu^\nu^\alpha^\beta}\tsr{F}{_\alpha_\beta}+\frac{1}{2}\tsr{\epsilon}{^\mu^\nu^\alpha^\beta}\tsr{F}{_\mu_\nu}\frac{\delta \tsr{\tilde F}{_\alpha_\beta}}{\delta A_\rho}\right]\\
	&= -\frac{a}{4}\int \dd^4x\Phi\left[2\partial_\mu\frac{\delta A_\nu}{\delta A_\rho}\tsr{\tilde F}{^\mu^\nu}+2\tsr{\tilde F}{^\alpha^\beta}\partial_\alpha\frac{\delta A_\beta}{\delta A_\rho}\right]\\
	&= -a\int\dd^4x\Phi\partial_\mu\frac{\delta A_\nu}{\delta A_\rho}\tsr{\tilde F}{^\mu^\nu}\\
	&= -a\int\dd^4x\Phi\partial_\mu(\delta(x-x')\delta_\nu^\rho)\tsr{\tilde F}{^\mu^\nu}\\
	&= a\partial_\mu(\Phi\tsr{\tilde F}{^\mu^\nu})\delta_\nu^\rho\\
	&= a[\partial_\mu\tsr{\tilde F}{^\mu^\nu}+\Phi\partial_\mu\tsr{\tilde F}{^\mu^\nu}]\delta_\nu^\rho\\
	&= a\partial_\mu\Phi\tsr{\tilde F}{^\mu^\nu}\delta_\nu^\rho\\
\end{align}

therefore

\begin{equation}
  \partial_\mu\tsr{F}{^\mu^\nu} = -4\pi ca\tsr{\tilde F}{^\mu^\nu}\partial_\mu\Phi
\end{equation}

After this there was another question about finding the effective current resulting from Phi. The idea of this question was to use Maxwell's equations and fit the above result into it. i.e using

\begin{equation}
  \partial_\mu\tsr{F}{^\mu^\nu} = \frac{4\pi}{c}J^\nu
\end{equation}

This was basically the heavy part of the first part of the course.

\section{The 2nd part of the course}

We'll start from the bottom with the simplest ways to solve Maxwell's equations.

\subsection{Gauge transformations}

A gauge transformation is of the form

\begin{equation}
  A_\mu \rightarrow a_\mu + \partial_\mu\lambda
\end{equation}

The known gauge transformations are

\begin{align}
	A_0=0\\
	\partial_i A^i=0\\
	A_3=0\\
	\partial_\mu A^\mu=0
\end{align}

and they are known by the names temporal gauge, coloumb gauge, axial gauge, and lorentz gauge. In electrostatics we'll work with coulomb gauge most of the times. The poisson equation is 

\begin{equation}
  -4\pi\rho = \nabla^2\Phi
\end{equation}

and in more general form it is

\begin{equation}
  \nabla^2\Psi(x)=f(x)
\end{equation}

we want to solve it. the first solution is

\begin{equation}
  \Psi(x)=\frac{-1}{4\pi}\int dV'\frac{f(x')}{|x-x'|}
\end{equation}

this solution is sometimes called a solution in superposition. This is a solution for when the boundary conditions are that the function tends to 0 at infinity. We'll use this solution to show we can move from any gauge to coulomb gauge.

We start with

\begin{equation}
  \tilde A_\mu=A_\mu+\partial_\mu \lambda
\end{equation}

and want to show that

\begin{align}
  0 &= \partial_i \tilde A^i\\
  &= \partial_i (A^i+\partial^i\lambda)\\
  &= \partial_i A^i+\partial_i\partial^i \lambda\\
  &= \nabla\cdot\bold A + \nabla^2\lambda
\end{align}

therefore we must find that

\begin{equation}
  \nabla^2\lambda = -\nabla\cdot\bold A
\end{equation}

and so we can define

\begin{equation}
  \lambda = \frac{1}{4\pi}\int dV'\frac{\nabla\cdot\bold A}{|x-x'|}
\end{equation}

Next we have an exercise. the following is given

\begin{align}
	E=-\nabla\Phi && \rho=-\frac{-\nabla^2\Phi}{4\pi}
\end{align}

find the potential and electric field above and below an infinite plate that's uniformly charged with charge density $\sigma$.

We'll start with

\begin{equation}
  \rho = \sigma \delta(z)
\end{equation}

so we have

\begin{equation}
  \Phi(x) = \int dV' \frac{\rho}{|x-x'|}=\int dV'\frac{\sigma\delta(z)}{|x-x'|}
\end{equation}

we'll work in cylindrical coordinates and thus

\begin{equation}
  \Phi(x)=\int r'dr'd\theta'dz'\frac{\delta(z')\sigma}{\sqrt{(x-x')^2+(y-y')^2+(z-z')^2}}
\end{equation}

now since the potential cannot possibly depend on x and y, we can choose specific values for those coordinates, say 0, and get

\begin{align}
  Phi(x=0,y=0,z)&=\int r'dr'd\theta'dz' \frac{\delta(z')\sigma}{\sqrt{r'^2+(z-z')^2}}\\
  &= 2\pi\sigma\int r'dr'\frac{1}{\sqrt{r'^2+z^2}}\\
  &= \left. 2\pi\sigma\sqrt{r'^2+z^2}\right|_0^\infty\\
  &= 2\pi\sigma(\infty-|z|)
\end{align}

Which isn't physical. But that's because our system is infinitely large. We can use the freedom to choose what the zero of our potential is and find

\begin{equation}
  \Phi(z)=-2\pi\sigma|z|
\end{equation}

and thus the electric field is

\begin{equation}
  \bold E = \begin{cases}
  	2\pi\sigma & z>0\\
  	-2\pi\sigma & z<0
  \end{cases}
\end{equation}

Another way of solving the poisson equation is the method of images. We can use that method thanks to the uniqueness of the solution. There were a lot of drawings now so enjoy these photos:

\begin{figure}[h!]
	\centering
	\includegraphics[width=0.6\linewidth,angle=-90]{IMG_3856.jpeg}
	\includegraphics[width=0.6\linewidth,angle=-90]{IMG_3857.jpeg}
	\includegraphics[width=0.6\linewidth,angle=-90]{IMG_3859.jpeg}
	\includegraphics[width=0.6\linewidth,angle=-90]{IMG_3860.jpeg}
	\includegraphics[width=0.6\linewidth,angle=-90]{IMG_3861.jpeg}
\end{figure}
\clearpage
When using the image method we need to make sure that all the boundary conditions are properly accounted for. We also need to reflect the images around every other plane that didn't spawn them themselves.

\subsection{Green's functions}

\lecture[2 hours]{27}{05}{2025}

The motivation is that we want to solve this:

\begin{equation}
  \nabla^2\Phi=-4\pi\rho
\end{equation}

Which is the poisson equation, which is linear and non-homogeneous. Linear functions are good because we can just sum up their solutions. We'll exploit Greens functions to use this and avoid solving non-homogenous equations which suck. we can instead solve something like this:

\begin{equation}\label{green1}
  \nabla^2G=\delta(x-x')
\end{equation}

then glue the two homogenous solutions together. We'll add one small addition to equation \ref{green1}. Green's functions are an operator of the geometry and only it. We'll also need to know a boundary condition, either the function's value on some area or its derivative.

Several examples of linear differential operators are the laplacian, divergence, and rotor. For these operators, and others, we can say that

\begin{equation}
  \hat L(af+bg)=a\hat Lf+b\hat Lg
\end{equation}

Now we look at the following:

\begin{align}
  \hat Lf = F(x) && \left.f\right|_{\partial V}=f_0(x)
\end{align}

How do we solve this? we define

\begin{align}
	\hat LG=\delta && f=G\star F=\int F(x')G(x,x')dx'+\int\limits_{\partial V}
\end{align}

Green's functions as we see are a function of two parameters, x and x'. One of them references the area we sample, the other the physical parameters we're sampling for. Now we do an exercise. Most of this session won't be very physics heavy, more math.

\subsubsection{an exercise}

A particle with mass m moving at velocity v(t) such that $v(t=0)=c$. The particle is affected by a drag force $F_{drag}=-\alpha v$ and another external force. \begin{enumerate}
	\item Write down the equations of motion
	\item Assume the external force is instantaneous and write down the equations of motion
	\item Write down Green's function
	\item Solve the equation of motion for a general external force
\end{enumerate}

1. The differential operator we want to solve here is Newton's equation $\sum F=ma$. therefore

\begin{align}\label{green2}
	F_{ext}-\alpha v=m\dot v
\end{align}

2. This means our force is

\begin{equation}
  F_{ext} = f_0\delta(t-t')/\tau
\end{equation}

where $\tau=\frac{\alpha}{m}$. We insert this into equation \ref{green2} and rearrange getting

\begin{equation}
  m\dot v + \alpha v=F_0\delta(t-t')/\tau
\end{equation}

3. For this we first want to find our L:

\begin{equation}
  \hat L = m\diffp*{}{t}+\alpha
\end{equation}

therefore

\begin{equation}
  \hat LG(t,t')=\delta(t-t')
\end{equation}

and our boundary condition is

\begin{equation}
  \left.G\right|_{t=0}=0
\end{equation}

The way to solve this is by splitting it up. For $t>t'$ we get:

\begin{equation}
  m\diffp*{G}{t}+\alpha G=0
\end{equation}

rearranging

\begin{equation}
  \frac{1}{G}\diffp{G}{t}=-\frac{\alpha}{m}
\end{equation}

therefore the solution is

\begin{equation}
  G=Ae^{-\frac{\alpha}{m}t}
\end{equation}

and trivially extending to both sides we get

\begin{equation}
G=
  \begin{cases}
  	Ae^{-\frac{\alpha}{m}t} & t<t'\\
  	Be^{-\frac{\alpha}{m}t} & t>t'
  \end{cases}
\end{equation}

and using the boundary condition we find

\begin{equation}
  G(t=0)=0 \rightarrow A=0
\end{equation}

notice we didn't eliminate B because our times are positive. We can find the other constant by inserting G into the original equation. We thus get:

\begin{equation}
  \alpha G = \alpha Be^{-\frac{\alpha}{m}(t-t')}\Theta(t-t')
\end{equation}

therefore

\begin{equation}
  m\diffp*{G}{t}=-B\alpha e^{-\frac{\alpha}{m}(t-t')}\Theta(t-t')+mBe^{-\frac{\alpha}{m}(t-t')}\delta(t-t')
\end{equation}

after some algebra we get

\begin{equation}
  mBe^{-\frac{\alpha}{m}(t-t')}\delta(t-t')=\delta(t-t')\frac{F_0}{\tau}
\end{equation}

therefore the following is true

\begin{align}
  &\lim\limits_{\epsilon\rightarrow0^+}\int\limits_{t'-\epsilon}^{t'+\epsilon}mBe^{-\frac{\alpha}{m}(t''-t')}\delta(t''-t')\dd t''\\
  &= \lim\limits_{\epsilon\rightarrow0^+}\int\limits_{t'-\epsilon}^{t'+\epsilon}\delta(t''-t')\frac{F_0}{\tau} \dd t''
\end{align}

therefore we must find that

\begin{equation}
  B=\frac{F_0}{\tau}\frac{1}{m}=\frac{f_0}{\alpha}
\end{equation}

and thus 

\begin{equation}
  G = \frac{1}{\alpha}F_0e^{-\frac{\alpha}{m}(t-t')}\Theta(t-t')
\end{equation}

4. To solve for a general force we use a convolution.

\begin{equation}
  v(t)=\int\limits_0^t\dd t'' G(t'',t')F_{ex}(t'')\cdot C_0
\end{equation}

where $C_0$ is just to make sure the units are right.

Next we look at the following kind of equation:

\begin{equation}
  \nabla_{x'}^2G(x,x')=-4\pi\delta^{(3)}(x-x')
\end{equation}

whose general solution is of the form

\begin{equation}
  G(x,x')=\frac{1}{|x-x'|}+g(x,x')
\end{equation}

such that 

\begin{equation}
  \nabla_{x'}^2g=0 \; \forall x' \in V
\end{equation}

And the most general solution for a potential is

\begin{equation}
  \Phi(x) = \int\limits_V \dd^3x'\rho(x')G(x,x')+\frac{1}{4\pi}\oint\limits_V da'\left[G(x,x')\diffp{\Phi}{n'}-\Phi(x')\diffp{G(x,x')}{n'}\right]
\end{equation}

Dirichlet boundary conditions are when the potential is given on the boundary, i.e that

\begin{equation}
  \left.G_D\right|_{\partial V} = 0
\end{equation}

Neumann boundary conditions are when the electric field on the boundary is given.

Now, an exercise.

\subsubsection{Another exercise}

An infinite plate is placed on the XY plane and held at a constant potential $\Phi(r)=V\Theta(a-r)$. Find an integral expression for the potential above the plate and develop it explicitly for along the symmetry axis.

We know the Green's function for an infinite plane:

\begin{align}
	\nabla^2G_D=-4\pi\delta(x-x') && G_D(x,y,z=0)=0
\end{align}

which gives us the following

\begin{equation}
  G_D=\frac{1}{\sqrt{(x-x')^2+(y-y')^2+(z-z')^2}}-\frac{1}{\sqrt{(x-x')^2+(y-y')^2+(z+z')^2}}
\end{equation}

therefore

\begin{equation}
\begin{aligned}
	\Phi(\bold x) &= \int\limits_{z>0}\rho(x')G_D(x,x')\dd x'-\frac{1}{4\pi}\int\dd x'\dd y'\Phi(x')\diffp{G}{n'}\\
  &= -\frac{1}{4\pi}\int\dd x'\dd y'\Phi(x')\diffp{G}{n'}\\
  &= \frac{1}{4\pi}\int\dd x'\dd y'\Phi(x')\diffp{G}{z'}\\
  &=[z'=0]= -\frac{1}{4\pi}\int\dd x'\dd y'\Phi(x')\frac{2z}{((x-x')^2+(y-y')^2+z^2)^{3/2}}\\
  &= \frac{z}{2\pi}\int \dd x' \dd y' \frac{\Phi(x',y')}{((x-x')^2+(y-y')^2+z^2)^{3/2}}
\end{aligned}
\end{equation}

and if we look along the axis of symmetry, such that $x=y=0$, we get:

\begin{equation}
  \begin{aligned}
  	\Phi(z) &= \frac{z}{2\pi}\int \dd x' \dd y' \frac{\Phi(x',y')}{(x'^2+y'^2+z^2)^{3/2}}\\
  	&= \frac{z}{2\pi}\int\limits_0^a r' \dd r' \int\limits_0^{2\pi}\dd \theta'\frac{V}{(r'^2+z^2)^{3/2}}\\\
  	&= V\left(1-\frac{z}{\sqrt{a^2+z^2}}\right)
  \end{aligned}
\end{equation}

That's the end of the exercise.

Next we look at a Direchlet Green's function, i.e

\begin{align}
	\left.G_D(x,x')\right|_{x'\in \partial V}=0 && G_D(x,x')=G_D(x',x)
\end{align}

and recall one of green's identities:

\begin{equation}
  \int \dd^3x\left(\phi\nabla^2\psi-\psi\nabla^2\phi\right) = \oint\limits_{\partial V}\dd a\hat n\cdot\left(\phi\nabla\psi-\psi\nabla\phi\right)
\end{equation}

Now let's define 

\begin{align}
	\phi(y)=G_D(x,y) && \psi(y)=G_D(x',y)\\
	\left.\phi\right|_{y\in \partial V}=0 && \left.\psi\right|_{y\in \partial V}=0
\end{align}

so we can write

\begin{align}
	&\int \dd^3y\left(G_D(x,y)\nabla^2G_D(x',y)-G_D(x',y)\nabla^2G_D(x,y)\right)\\
	&= \oint\limits_{\partial V}\dd a\hat n \left(G_D(x,y)\nabla G_D(x',y)-G_D(x',y)\nabla G_D(x,y)\right)
\end{align}

from here there was another transition I didn't have time to write down because she sped through it in a moment and we were going to take the stat-mech midterm.
\\
\\
\\
\lecture[2 hours]{03}{06}{2025}

The laplace problem as we remember is an equation of the form

\begin{align}
  \nabla^2\Phi(x)=0 && x\in V
\end{align}

with given boundary conditions. Separation of variables is a method for solving laplace problems. Another way is Greens' functions, but we'll focus on separation today. When solving using separation of variables we will usually split the potential up as:

\begin{equation}
  \Phi(\bold x) = \sum\limits_{n=1}^\infty a_m U_n(\bold x)
\end{equation}

where U has the following properties:

\begin{enumerate}
	\item \begin{equation}
		\nabla^2U(\bold x)=0
	\end{equation}
	\item Orthonormality: \begin{equation}
		\int\limits_a^b U_n^*(\bold x)U_m(\bold x)\dd x = \delta_{n,m}
	\end{equation}
	\item A whole orthonormal system \begin{equation}
		\sum\limits_n^\infty U^*_n(\bold x')U_n^*(\bold x) = \delta(\bold x'-\bold x)
	\end{equation}
\end{enumerate}

using the last two properties we can extract the coefficients:

\begin{equation}
  a_n = \int\limits_a^b \Phi(x)U^*_n(\bold x)
\end{equation}

\subsection{Separation of variables in the laplace equation in cartesian coordinates}

\subsubsection{example}

The following is the most common example for this:

\begin{equation}
  U_n^*(\bold x) = \sqrt{\frac{2}{L}}\sin\left(\frac{\pi nx}{L}\right)
\end{equation}

where $0<x<L$. If we then check orthonormality we find

\begin{align}
	\frac{2}{L}\int\limits_0^L\sin\left(\frac{\pi nx}{L}\right)\sin\left(\frac{\pi mx}{L}\right)\dd x = \delta_{n,m}
\end{align}

and checking completeness we find

\begin{equation}
	\frac{2}{L}\sum\limits_{n=1}^\infty \sin\left(\frac{\pi nx}{L}\right)\sin\left(\frac{\pi mx}{L}\right) = \delta(x-x')
\end{equation}

\subsubsection{A recipe for solving in cartesian coordinates}

This likely works in all coordinates but we work in cartesian. The equation we have is

\begin{equation}
  \nabla^2\Phi(x,y,z)=0
\end{equation}

inside a box:

\begin{equation}
  0<x^i<a^i
\end{equation}

For example we will work with homogenous dirichlet boundary conditions:

\begin{align}
  \Phi(\bold x)=0 && \forall x^i=0,a^i, i\in \{1,2\}
\end{align}

We'll guess a solution of the form

\begin{equation}
  \Phi(\bold x) = X(x)Y(y)Z(z)
\end{equation}

and insert it into our laplace problem, and find

\begin{equation}\label{Ansatz_solution}
  \frac{X''}{X}(x)+\frac{Y''}{Y}(y)+\frac{Z''}{Z}(z)=0
\end{equation}

and since our right side is constant, each of the fractions must also be constant. (If this is confusing, refer to your PDE course for a proof). Therefore we can write

\begin{align}
	\frac{X''}{X}=-\alpha^2 && \frac{Y''}{Y}=-\beta^2 && \frac{Z''}{Z}=-\gamma^2
\end{align}

We choose the constants to be negative squares because it's convenient. Again refer to a PDE course for a better reason. Inserting into equation \ref{Ansatz_solution} we find

\begin{equation}
  \alpha^2+\beta^2+\gamma^2=0
\end{equation}

noting that all three of them can be complex numbers. There are two kinds of solutions, depending on whether the constant is 0 or not. If for example $\alpha \neq 0$ then

\begin{equation}
  X_\alpha(x^1)=A^1e^{i\alpha x^1}+B^1e^{-i\alpha x^1}
\end{equation}

but if $\alpha =0$ then

\begin{equation}
  X_0(x^1)=A^1x^1+B^1
\end{equation}

This is of course identical for $\beta$ and $\gamma$. We can find the coefficients A and B using the boundary conditions. In our specific example, the boundary conditions limit $\alpha$ and $\beta$ too. We find that:

\begin{align}
  \alpha_n=\frac{\pi n}{a^1} && \beta_m=\frac{\pi m}{a^2}
\end{align}

and the functions themselves will be

\begin{align}
	X_n(x)=\sin\left(\frac{\pi n x}{a^1}\right) && Y_m(y) = \sin\left(\frac{\pi mx}{a^2}\right)
\end{align}

This is missing orthonormality factors but the general idea is clear. Since we have no boundary conditions for Z the solution is

\begin{equation}
  \Phi(x,y,z)=\sum\limits_{n,m} Z_{n,m}(z)X_n(x)Y_n(y)
\end{equation}

where 

\begin{equation}
  \gamma^2_{n,m}=-(\alpha_n^2+\beta_m^2)
\end{equation}

For example, if $\gamma_{n,m} \neq 0$ then 

\begin{equation}
  Z_{n,m}(z) = A_{n,m}e^{i\gamma_{n,m}z}+B_{n,m}e^{-i\gamma_{n,m}z}
\end{equation}

and if $\gamma_{n,m}=0$ then

\begin{equation}
  Z_{n,m}(z) = A_{n,m}z+B_{n,m}
\end{equation}

We'll find Z by spanning it:

\begin{align}
	Z_{n,m}(z) = \frac{2}{a^1}\frac{2}{a^2}\int\limits_0^{a^1}\dd x \int\limits_0^{a^2}\dd y \Phi(x,y,z)\sin\left(\frac{\pi nx}{a^1}\right)\sin\left(\frac{\pi my}{a^2}\right) && n,m \in \mathbb N
\end{align}

We'll then insert $z=0,a^i$ and find the As and Bs using the boundary conditions. This should be familiar from either your PDE or Fourier course.

\clearpage
\subsubsection{An example}

\begin{figure}[!h]
	\centering
	\includegraphics[width=0.6\linewidth,angle=-90]{IMG_3908.jpeg}
\end{figure}

A two dimensional problem is given. Find the potential in the area $0<x<a$ and $y>0$ where the boundary conditions are $\Phi=V$ at $y=0$ and $\Phi = 0$ at $x=0,a$

\subsubsection{Solution}

We'll write down our solution guess:

\begin{equation}
  \Phi(x,y) = X(x)Y(y)
\end{equation}

and find

\begin{equation}
  \frac{1}{X}\diff[2]{X}{x}+\frac{1}{Y}\diff[2]{Y}{y}=0
\end{equation}

which gives us the following:

\begin{equation}
  \frac{1}{X}\diff[2]{X}{x} = -\alpha^2 = \frac{1}{Y}\diff[2]{Y}{y}
\end{equation}

with $\alpha \in \mathbb R \backslash \{0\}$ thus the general solution is

\begin{align}
	X(x) = A\sin(\alpha x)+B\cos(\alpha y)
\end{align}

and from the boundary condition $X(0)=0$ we find $B=0$. We have another boundary condition to use: $X(a)=0$ which gives us $\alpha_n=\frac{\pi n}{a}$. therefore:

\begin{equation}
  X(x) = \sum\limits_{n=1}^\infty A_n\sin\left(\frac{\pi nx}{a}\right)
\end{equation}

And as for Y, the general solution is of the form of two exponentials

\begin{equation}
  Y(y) = Ce^{\alpha y}+De^{-\alpha y}
\end{equation}

but because the potential must go to 0 at infinity, we find the $C=0$ and thus

\begin{equation}
  \Phi(x,y) = \sum\limits_{n=1}^\infty A_ne^{-\frac{\pi ny}{a}}\sin\left(\frac{\pi nx}{a}\right)
\end{equation}

and from $\Phi(x,0)=V$ we find

\begin{equation}
  V=\sum\limits_{n=1}^\infty A_n\sin\left(\frac{\pi nx}{a}\right)
\end{equation}

and using the completeness we find

\begin{equation}
  A_n = \frac{2}{a}V\int\limits_0^a\dd x\sin\left(\frac{\pi nx}{a}\right)=\frac{4V}{\pi n}\begin{cases}
  	1 & n=2k+1\\0 & n=2k
  \end{cases}
\end{equation}

Therefore

\begin{equation}
  \Phi(x,y) = \frac{4V}{\pi}\sum\limits_{k=1}^\infty\frac{1}{2k-1}e^{-(2k-1)\frac{\pi y}{a}}\sin\left(\frac{\pi (2k-1) x}{a}\right)
\end{equation}

\subsection{Separation of variables in the laplace equation in spherical coordiantes}

We can write the potential as

\begin{equation}
  \Phi(r,\theta,\varphi)=R(r)P(\theta)Q(\varphi)
\end{equation}

The form of $R_l(r)$ is

\begin{equation}
  R_l(r)=Ar^l+Br^{-(l+1)}
\end{equation}

The form of $Q_m(\varphi)$ is

\begin{equation}
  Q_m(\varphi) = Ce^{im\varphi}+De^{-im\varphi}
\end{equation}

The form of $P_l^m(x=\cos\theta)$ are Legendre polynomials

\begin{align}
	l=0,1,2,... && m\leq l=0,\pm 1, \pm 2, ..., \pm l
\end{align}

\subsubsection{Azimuthal symmetry}

A problem has azimuthal symmetry if $\varphi \rightarrow \varphi+\varphi_0$ is a symmetry. In this case the functions $P_l(x)=P_l^0(x)$ are called legendre polynomials and they satisfy:

\begin{itemize}
	\item Orthogonality \begin{equation}
		\int\limits_{-1}^1\dd x P_l(x)P_{l'}(x) = \frac{2}{2l+1}\delta_{l,l'}
	\end{equation}
	\item completeness \begin{equation}
		\sum\limits_{l=0}^\infty \frac{2l+1}{2}P_l(x)P_l(y)=\delta(x-y)
	\end{equation}
	\item even-ness \begin{equation}
		P_l(x) = (-1)^lP_l(-x)
	\end{equation}
	\item Useful identity \begin{equation}\label{legendre_identity}
		\frac{1}{|\bold r-\bold r'|}=\frac{1}{r_>}\sum\limits_{l=0}^\infty \left(\frac{r_<}{r_>}\right)^lP_l(x)
	\end{equation} where $r_<=min(\bold r, \bold r')$ and $r_>=max(\bold r, \bold r')$ and $x=\hat r\cdot \hat {r'}$.
\end{itemize}

The first Legendre polynomials are

\begin{align}
	P_0(x) &= 1\\
	P_1(x) &= x\\
	P_2(x) &= \frac{1}{2}(3x^2-1)\\
	P_3(x) &= \frac{1}{2}(5x^3-3x)
\end{align}

\subsubsection{example}

\begin{figure}[h!]
	\centering
	\includegraphics[width=0.6\linewidth,angle=0]{IMG_3909.jpeg}
\end{figure}


Find the electric potential of a ring or radius R which is charged with a uniform charge density $\lambda$. The ring's axis coincides with the z axis and it's located at the height h above the XY plane. \emph{We use the convention that charges are always located at coordinates denoted with ', and the area we're solving for does not have a '.}

\subsubsection{solution}

Along the axis of symmetry:

\begin{equation}
  \Phi(z) = \int\limits_0^{2\pi}\frac{\lambda R}{\sqrt{R^2+(z-h)^2}}\dd\varphi = \frac{2\pi \lambda R}{\sqrt{R^2+(z-h)^2}}
\end{equation}

We can rewrite the denominator as

\begin{equation}
  \frac{1}{\sqrt{R^2+(z-h)^2}} = \frac{1}{|z\cdot\hat z-\bold{r'}|}
\end{equation}

where

\begin{equation}
  \hat z\cdot \hat r=\cos(\theta')=\frac{h}{\sqrt{h^2+R^2}}
\end{equation}

We did this because of the identity we have for legendre polynomials in equation \ref{legendre_identity}, which we use to find:

\begin{equation}
  \frac{1}{\sqrt{R^2+(z-h)^2}} = \sum\limits_{l=0}^\infty \frac{r_<^l}{R_>^{l+1}}P_l(\cos(\theta'))
\end{equation}

On the other hand we saw in the lecture that a general solution for a legendre equation in spherical coordinates with azimuthal symmetry is of the form

\begin{equation}
  \Phi(r,\theta) = 2\pi\lambda R \sum\limits_{l=0}^\infty \begin{cases}
  	A_l r^{-l-1}P_l(\cos(\theta)) & r>r'\\
  	B_l r^l P_l(\cos(\theta)) & r<r'
  \end{cases}
\end{equation}

and along the z axis we have $P_l(cos(\theta)=1)=1$ therefore we can compare powers of r and find

\begin{align}
	A_l = {r'}^lP_l(cos(\theta')) && B_l = \frac{1}{{r'}^{l+1}}P_l(\cos(\theta'))
\end{align}

\clearpage
\subsection{Problems without azimuthal symmetry}
\lecture[2 hours]{10}{06}{2025}

In this case the solution will be the Legendre equation with $m\neq 0$. In that case, the solutions are the associated legendre polynomials:

\begin{equation}
  \tsr{P}{_l^m}(x) = \begin{cases}
  	(-1)^m(1-x^2)^{m/2}\diff{P_l}{x} & m \geq 0\\
  	(-1)^m\frac{(l+m)!}{(l-m)!}\tsr{P}{_l^{-m}}(x) & m<0
  \end{cases}
\end{equation}

They are orthogonal:

\begin{equation}
  \int\limits_{-1}^1 \tsr{P}{_l^m}\tsr{P}{_{l'}^m}=\frac{2}{2l+1}\frac{(l+m)!}{(l-m)!}\delta_{l,l'}
\end{equation}

And they are even

\begin{equation}
  \tsr{P}{_l^m}(-x) = (-1)^{l+m}\tsr{P}{_l^m}(x)
\end{equation}

Spherical harmonics are

\begin{equation}
  Y(\theta,\varphi) = \sqrt{\frac{2l+1}{4\pi}\frac{(l-m)!}{(l+m)A!}}\tsr{P}{_l^m}(\cos\theta)e^{im\varphi}
\end{equation}

Where $-l\leq m\leq l$. The first spherical harmonic is

\begin{equation}
  Y_{0,0}(\theta\varphi)=\frac{1}{\sqrt{4\pi}}
\end{equation}

The second spherical harmonic (ignoring the idea you can't really order them) is

\begin{equation}
  Y_{1,0}(\theta,\varphi) = \sqrt{\frac{3}{4\pi}}\cos\theta
\end{equation}

and

\begin{equation}
  Y_{1,\pm1}(\theta,\varphi) = \mp\sqrt{\frac{3}{8\pi}}\sin\theta e^{\pm i\varphi}
\end{equation}

then

\begin{equation}
  Y_{2,0}(\theta,\varphi) = \sqrt{\frac{5}{16\pi}}(3\cos^2\theta-1)
\end{equation}

These are orthonormal

\begin{equation}
  \oiint \dd\Omega Y^\star_{lm}(\hat r)Y_{l',m'}(\hat r) = \delta_{l,l'}\delta_{m,m'}
\end{equation}

And complete

\begin{equation}
  \sum\limits_{l=1}^\infty \sum\limits_{m=-l}^l Y^\star_{lm}(\theta,\varphi)Y_{lm}(\theta',\varphi')=\delta(\varphi-\varphi')\delta(\theta-\theta')\frac{1}{\sin\theta}
\end{equation}

\subsubsection{Exercise}

A spherical shell of radius R is held at the potential $\Phi(\theta,\varphi)=V\sin\theta\sin\varphi$. Find the potential in all space.

\subsubsection{solution}

If we forgot how to use Green's functions, we can solve this by writing down the most general solution using harmonics then restrict it. The general solution is

\begin{equation}
  \Phi=\sum\limits_{l=0}^\infty\sum\limits_{m=-l}^l\left[A_{lm}r^l+B_{lm}r^{-(l+1)}\right]
	Y_{lm}(\theta,\varphi)
\end{equation}

We then separate space into two parts: $r>R$ and $r<R$. Inside the shell we have

\begin{equation}
  \Phi = \sum\limits_{l=0}^\infty\sum\limits_{m=-l}^l A_{lm}r^lY_{lm}(\theta,\varphi)
\end{equation}

and outside the shell we have
\begin{equation}
  \Phi = \sum\limits_{l=0}^\infty\sum\limits_{m=-l}^lB_{lm}r^{-(l+1)}Y_lm
\end{equation}

We'll stitch the potential at $r=R$ with the following conditions for it to be continuous:

\begin{equation}
  Y_{lm}A_{lm}R^l = B_{lm}R^{-(l+1)}Y_{lm}
\end{equation}

and since the potential is known then

\begin{equation}
  V\sin\theta\sin\varphi = \sum\limits_{l=0}^\infty\sum\limits_{m=-l}^lA_{lm}R^lY_{lm}
\end{equation}

therefore

\begin{align}
	Y_{11} = -\frac{3}{8\pi}\sin\theta e^{i\varphi} && Y_{1,-1}=\frac{3}{8\pi}\sin\theta e^{-i\varphi}
\end{align}

but on the other hand

\begin{equation}
  V\sin\theta\sin\varphi = Vi\frac{2\pi}{3}\left(Y_{11}-Y_{1,-1}\right)
\end{equation}

therefore

\begin{align}
  A_{1,1}=i\sqrt{\frac{2\pi}{3}}\frac{V}{R} && A_{1,-1} = -i\sqrt{\frac{2\pi}{3}}\frac{V}{R}
\end{align}

Therefore, inside the sphere:

\begin{equation}
  \Phi(r,\theta,\varphi) = iV\sqrt{\frac{2\pi}{3}}\frac{r}{R}Y_{1,1}-iV\sqrt{\frac{2\pi}{3}}\frac{r}{R}Y_{1,-1} = V\sin\theta\sin\varphi\frac{r}{R}
\end{equation}

And outside the sphere:

\begin{equation}
  \Phi(r,\theta,\varphi) = iV\sqrt{\frac{2\pi}{3}}\frac{R^2}{r^2}Y_{1,1}-iV\sqrt{\frac{2\pi}{3}}\frac{R^2}{r^2}Y_{1,-1} = V\sin\theta\sin\varphi\frac{R^2}{r^2}
\end{equation}

The translation from Ys to trig functions being done using the identities for exponent sums.

\section{Multipoles}

We can discuss a multipole expansion when we have a set of charges that's within one scale of distances apart and we're looking at it from a distance that is much larger than the distance scale within the charge distribution. It originates from the superposition of the potentials of all those charges, and looks as such:

\begin{equation}
  \frac{1}{|x-x'|}=\frac{1}{|x|}+\frac{x'_ix_i}{|x|^3}+\frac{1}{2}\frac{x'_ix_j}{|x|^5}(3x_ix_j-\delta_{ij}|x|^2)+...
\end{equation}

So

\begin{equation}
  \Phi(x) = \frac{q}{|x|}+\frac{x\cdot d}{|x|^3}+Q_{ij}\frac{x_ix_j}{|x|^5}
\end{equation}

Where

\begin{align}
	q = \int\dd^3x'\rho(x') && d=\int\dd^3x'\rho(x')x' && Q=\frac{1}{2}\int\dd^3x'\rho(x')(3x'_ix'_j-r'^2\delta_{ij})
\end{align}

And Q with further indices are multipoles of higher orders.

The monopole is a 0th order tensor, the dipole is a 1st order tensor, the quadropole is a 2nd order tensor, etc.

\subsection{exercise}

Calculate the leading moment of a ring with radius R with charge density $\lambda = \lambda_0\cos\varphi$.

The monopole is

\begin{equation}
  q = \int\limits_0^{2\pi} \lambda_0\cos\varphi\dd\varphi = 0
\end{equation}

The dipole moment is

\begin{align}
  d &= \int\dd^3r'\rho(r')r' \\
  &= \int\dd^3r' \lambda_0\cos\varphi\delta(r-R)\delta(z)r'\\
  &=\int\dd^3r' \lambda_0\cos\varphi\delta(r-R)\delta(z) (R\cos\varphi\hat x+R\sin\varphi\hat y)\\
  &= \int\limits_0^{2\pi}\dd \varphi \lambda_0\cos\varphi R^2(\cos\varphi\hat x+\sin\varphi\hat y)\\
  &= \lambda_0\pi R^2 \hat x
\end{align}

And if the exercise was about a disk rather than a ring, charged with surface charge $\sigma_0$ then

\begin{equation}
  \rho(r) = \sigma_0\Theta(a-r)\delta(z)
\end{equation}

and the monopole moment is

\begin{align}
	q = \int\dd^3x\rho(x) = \int\limits_0^{2\pi}\int\limits_{-\infty}^\infty\int\limits_0^a \sigma_0\delta(z)r\dd r\dd z\dd \theta = \sigma_0 a^2\pi
\end{align}

The duopole is equal to 0, and the quadropole is

\begin{equation}
  Q_{ij} = \frac{1}{2}\int(3x'_ix'_j-r'^2\delta_{ij})\rho(x')\dd^3x'
\end{equation}

so

\begin{align}
	Q_{xx} &= \frac{1}{2}\int(3x'^2-r'^2)\rho(x')\dd^3x' =...=\\
	&= \frac{\sigma_0}{2}\int\limits_0^{2\pi}\int\limits_0^a(3r'^2\cos^2\varphi'-r'^2)r'\dd r' \dd \varphi'\\
	&= \sigma_0\frac{a^4\pi}{8}
\end{align}

from symmetry we get

\begin{equation}
  Q_{yy} = \sigma_0\frac{a^4\pi}{8}
\end{equation}

and because $Q_{xx}+Q_{yy}+Q_{zz}=0$:

\begin{equation}
  Q_{zz} = -(Q_{xx}+Q_{yy})
\end{equation}

and for $Q_{ij}$ where $i\neq j$:

\begin{equation}
  Q_{ij} = Q_{xy} = 0
\end{equation}

because

\begin{equation}
  Q_{ij} \propto \int\sin\varphi\cos\varphi\dd\varphi=0
\end{equation}

Therefore

\begin{equation}
  \Phi(x) = \sigma_0\pi a^2\left[\frac{1}{r} + \frac{\frac{a^2}{4}(x^2+y^2-2z^2}{2r^5}\right]
\end{equation}

\emph{The 2 following trainings were not typed out since they happened on Zoom due to the war. The next training is session 12.}

\lecture[2 hours]{01}{07}{2025}

\section{Magnetic stuff}
\subsection{Simple magnetic materials}

Simple magnetic materials are materials where we assume the reaction to magnetization is linear, i.e

\begin{equation}
  \bold M = \chi_m\bold H
\end{equation}

where $\chi_m$ is the magnetic susceptibility. We also know that

\begin{equation}
  \bold B=\bold H+4\pi \bold M=(1+4\pi\chi_m)\bold H=\mu \bold H
\end{equation}

\subsubsection{exercise}

A ball of radius a and permeability coefficient $\mu$ is placed in a constant magnetic field $\bold B = B\hat z$. Find B and H in all space.

We begin by noticing that there are no free currents in the problem. This means

\begin{equation}
  \bold\nabla\times \bold H = 0
\end{equation}

Therefore H can be written as a gradient of a function

\begin{equation}
  \bold H = -\bold\nabla\phi_m
\end{equation}

or 

\begin{equation}
  -\nabla\cdot\bold H = \nabla^2\phi_m =0
\end{equation}

where $\phi_m$ is the dual potential, since the field H is constant and thus it has 0 divergence. Now we use separation of variables, and since there is azimuthal symmetry we use Legendre polynomials. We'll split the solution for two zones, inside and outside the sphere.

\begin{align}
	\phi_m(r<a)&=\sum\limits_{l=0}^\infty A_lr^lP_l(\cos\theta)\\
	\phi_m(r>a)&=\sum\limits_{l=0}^\infty B_lr^lP_l(\cos\theta)+C_lr^{-(l+1)}P_l(\cos\theta)\label{265}
\end{align}

And we have the following boundary conditions:

\begin{enumerate}
	\item The potential is finite in the origin, thus $D_l=0$
	\item The field at infinity
	\item The potential is continuous at the edge of the sphere
	\item $\Delta B_\perp=0$ (is always correct)
\end{enumerate}

Now we insert these conditions. 1 was already used. As for 2:

\begin{align}
	H_z=B_0=-\diffp*{\phi_m}{z}
\end{align}

therefore

\begin{equation}
  \phi_m=-B_0z=-B_0r\cos\theta
\end{equation}

next by comparing coefficients with equation \ref{265} we find

\begin{align}
	B_1=-B_0
\end{align}

Next we use point 3:

\begin{align}
	\sum\limits_l\left[B_la^l+C_la^{-(l+1)}\right]P_l(\cos\theta)=\sum A_la^lP_l(\cos\theta)
\end{align}

which gives us

\begin{align}
	\begin{cases}
		A_1a=C_1a^{-2}-B_0a & l=1\\
		A_la^l=C_la^{-(l+1)} \rightarrow C_l=A_la^{2l+1} & l \neq 1
	\end{cases}
\end{align}

and using the last boundary condition, by differentiating with respect to r because we're looking for the jump in the field in the radial direction,

\begin{align}
	B(r=a^-) &=-\mu\sum\left[lA_la^{l-1}\right]P_l(\cos\theta)\\
	B(r=a^+)&=-\sum \left[\tilde B_lla^{l-1}-C_l(l+1)a^{-(l+2)}\right]P_l(\cos\theta)
\end{align}

and we demand

\begin{equation}
  B(r=a^-)=B(r=a^+)
\end{equation}

at this point we're effectively done, we just need to fit all the puzzle pieces together and differentiate the potential we get to find the field. The field is, skipping algebra,

\begin{align}
  \phi_m(r<a) &= -\frac{3B_0}{\mu+2}r\cos\theta\\
  \phi_m(r>a) &= -B_0r\cos\theta+B_0\left(\frac{\mu-1}{\mu+2}\right)\frac{a^3}{r^2}\cos\theta
\end{align}

which gives us the field

\begin{equation}
  \bold H = -\nabla\phi_m= \begin{cases}
  	\frac{3B_0}{\mu+2}\hat z & r<a\\
  	B_0\hat z+B_0\left(\frac{\mu-1}{\mu+2}\right)\frac{a^3}{r^2}(2\cos\theta\hat r + \sin\theta\hat \theta) & r>a
  \end{cases}
\end{equation}

and

\begin{equation}
  \bold B = \mu \bold H= \begin{cases}
  	\frac{\mu}{\mu+2}B_0\hat z & r<a\\
  	B_0\hat z+B_0\left(\frac{\mu-1}{\mu+2}\right)\frac{a^3}{r^2}(2\cos\theta\hat r + \sin\theta\hat \theta) & r>a
  \end{cases}
\end{equation}

and the magnetization is

\begin{equation}
  \bold M = \frac{1}{4\pi}(\bold B-\bold H)=\frac{3}{4\pi}\frac{\mu -1}{\mu+2}B_0\hat z
\end{equation}

and the magnetic moment

\begin{equation}
  \bold m=\frac{4\pi}{3}a^3\bold M = \frac{\mu-1}{\mu+2}B_0a^3\hat z
\end{equation}

and thus we can write the field using the moment

\begin{equation}
  \phi_m=-B_0z+\begin{cases}
  	\frac{\bold m\cdot\bold r}{a^3} & r<a \\
  	\frac{\bold m\cdot \bold r}{r^3} & r>a
  \end{cases}
\end{equation}

\subsubsection{Limits of $\mu$:}

\begin{enumerate}
	\item If $\mu\rightarrow 0$ we get $\bold B=0$ and we call this a superconductor.
	\item If $0<\mu<1$ then $\bold B \parallel -B_0\hat z$ and we call this a diamagnet.
	\item If $\mu\rightarrow 1$ then the material is transparent for the magnetic field. We don't really have a name for these.
	\item If $1<\mu<\infty$ then $\bold B \parallel B_0\hat z$ and we call this a paramagnet.
	\item If $\mu\rightarrow\infty$ then $\bold H=0$ and we call this a ferromagnet.
\end{enumerate}

All these points are properties of the material in question and only apply inside the material.

\clearpage

\section{Radiation}

Radiation is change in time of the fields. These changes propagate at the speed of causality, which is the speed of light in vaccuum. We work in Lorentz gauge when working with radiation, $\partial_\mu A^\mu=0$, and we use the following maxwell equations

\begin{equation}
	\partial_\mu \tsr{F}{^\mu^\nu}=\frac{4\pi}{c}J^\mu
\end{equation}

and we use the following definition of the retarded time

\begin{equation}
  t_{ret}=t-\frac{|\bold x-\bold x'|}{c}
\end{equation}

with $\bold x$ being the location we're interested in or are located, and $\bold x'$ being the location of the event that happened.

The retarded potential is 

\begin{equation}
  A^4(\bold x,t) = \frac{1}{c}\int\limits_{V'}\frac{J^4(\bold x',t_{ret})}{|\bold x-\bold x'|}\dd V'
\end{equation}

Typical questions are of the kind where

\begin{enumerate}
	\item Things switching on and off (stuff starts moving and stops)
	\item Coulomb's law
	\item multiplole development
\end{enumerate}

\subsection{exercise}

A straight infinite neutral wire has current passing through it

\begin{equation}
  I(t) = \begin{cases}
  	0 & t\leq0\\
  	\kappa t & t>0
  \end{cases}
\end{equation}

Find the potential.

First we notice that since the wire is neutral, $A^0=0$. To solve the vector part we write down the current density, assuming the wire is placed along the z axis:

\begin{align}
	j=I(t)\delta(x)\delta(y)\hat z
\end{align}

We'll choose an observation point at $\bold x=(x,y,0)$ and we'll work in cylindrical coordinates.

\begin{align}
	\bold A(\bold x,t) &= \frac{1}{c}\int\frac{\bold j(\bold x',t_{ret})}{|\bold x-\bold x'|}\dd V'\\
	&= \frac{1}{c}\hat z\int\frac{I(t_{ret})\delta(x')\delta(y')}{\sqrt{(x-x')^2+(y-y')^2+z'^2}}\dd x'\dd y'\dd z'\\
	&= \frac{1}{c}\hat z\int\dd z' \frac{I(t_{ret})}{\sqrt{r^2+z'^2}}
\end{align}

and we remember that $I(t_{ret})>0$ happens at $t_{ret}>0$, i.e when

\begin{equation}
  t_{ret} = t-\frac{|x-x'|}{c}=t-\frac{\sqrt{r^2+z'^2}}{c} > 0
\end{equation}
 
 rewriting
 
 \begin{equation}
  t>\frac{\sqrt{r^2+z'^2}}{c}
\end{equation}

and again we find

\begin{equation}
  -\sqrt{(ct)^2-r^2} < z' < \sqrt{(ct)^2-r^2}
\end{equation}

which gives us our bounds for z', therefore

\begin{equation}
  =\frac{1}{c}\hat z \int\limits_{-\sqrt{(ct)^2-r^2}}^{\sqrt{(ct)^2-r^2}} \frac{\kappa t_{ret}}{\sqrt{r^2+z'^2}}\dd z'
\end{equation}

and eventually we find

\begin{equation}
  =\frac{2\kappa}{c^2}\left(ctln\frac{ct+\sqrt{(ct)^2-r^2}}{r}-\sqrt{(ct^2)-r^2}\right) \hat z
\end{equation}

and to find the fields we could then do

\begin{align}
	\bold B &= \nabla\times \bold A\\
	\bold E &= -\frac{1}{c}\diff{a}{t}
\end{align}

\subsubsection{Generalized Coulomb's law}

We have some charge q moving along some path $x_q^\mu(t)$. It's speed is $\bold v=\diff{\bold x}{t}$. We'll define

\begin{align}
  y^\mu&=(ct_{ret},\bold x_q(t_{ret})\\
  R^\mu&=x^\mu-y^\mu
\end{align}

We then have the following

\begin{align}
	\phi(\bold x,t) &= \left.\frac{qc}{||R|\cdot c-\bold R\cdot \bold v|}\right|_{t=t_{ret}}\\
	\bold A &= \left.\frac{q\bold v}{|R|c-\bold R\cdot\bold v}\right|_{t=t_{ret}}
\end{align}

\subsubsection{exercise}

A point charge e is moving in a circular path of radius d with constant angular velocity $\omega$. at $t=0$ it is on the x axis. Find the potential on the z axis.

The charge's position is

\begin{equation}
  \bold x_q = d(\cos(\omega t)\hat x + \sin(\omega t)\hat y)
\end{equation}

and our observation point will be $\bold r=z\hat z$. Therefore

\begin{equation}
  \bold R(t_{ret})=\bold r-\bold x_q=z\hat z-d\hat xcos(\omega t_{ret})-d\hat y\sin(\omega t_{ret})
\end{equation}

and

\begin{equation}
  |\bold R| = \sqrt{z^2+d^2}
\end{equation}

and

\begin{equation}
  \bold v(t=t_{ret}) = d\omega(-\sin(\omega t_{ret})\hat x+\cos(\omega t_{ret})\hat y)
\end{equation}

therefore

\begin{equation}
  \bold R \cdot \bold v = 0\hat z + d^2\omega\sin(\omega t)\cos(\omega t)-...=0
\end{equation}

since we're doing circular motion. Next

\begin{equation}
  \bold A(z, t) = \frac{ed\omega(-\sin(\omega t_{ret})\hat x+\cos(\omega t_{ret})\hat y}{c\sqrt{z^2+d^2}}
\end{equation}

where 

\begin{equation}
  t_{ret} = t-\frac{|R|}{c}=t-\frac{\sqrt{z^2+d^2}}{c}
\end{equation}

In general in questions of this kind the important thing is to properly find the path and just solve it.

\clearpage
\section{repetition training}

\lecture[2 hours]{08}{07}{2025}

\emph{Due to the war, the semester was cut short by a week, with week 13 being only for repetitions and reinforcing. Due to this, the next session does not include new material.}

\subsection{Lagrangian topic example}

The following action is given

\begin{equation}
  S[\phi]=\int\frac{1}{2}\partial_\mu\phi^\mu\partial^\mu\phi_\mu-\Lambda^\mu(1-\cos\frac{\phi}{f})\dd^4x
\end{equation}

where $\Lambda$ and f are constants and $\phi$ is a real scalar field.

\begin{enumerate}
	\item Develop the Lagrangian to leading order in $\phi<<f$
	\item Find the equations of motion
	\item develop it to the next order. Is the Lagrangian symmetric to reflections?
	\item develop it one more time. Is this the most general possible form?
	\item The following action si given: \begin{equation}
		S[\phi_1,\phi_2] = \int\frac{1}{2}\partial_\mu\phi_1\partial^\mu\phi_1+\frac{1}{2}\partial_\mu\phi_2\partial^\mu\phi_2-\Lambda^4\left(1-\cos\ins{\frac{\phi_1+\phi_2}{f}}\right)
	\end{equation} develop this to leading order. Is the Lagrangian symmetric to reflections? Write the equations of motion.
\end{enumerate}

\subsubsection{Lagrangian development}

Since 

\begin{equation}
  \cos\left(\frac{\phi}{f}\right) \approx 1-\ins{\frac{\phi}{f}}^2\cdot\frac{1}{2}
\end{equation}

then

\begin{equation}
  \lagr = \frac{1}{2}\partial_\mu\phi^\mu\partial^\mu\phi_\mu-\frac{\Lambda^4}{2}\ins{\frac{\phi}{f}}^2
\end{equation}

\subsubsection{Equations of motion}

We'll use the variation method. We will vary the field:

\begin{align}
	\phi \rightarrow \phi+\delta\phi\\
	\delta S = S[\phi+\delta\phi]-S[\phi]
\end{align}

therefore

\begin{align}
	\partial_\mu\phi^\mu\partial^\mu\phi_\mu &\rightarrow \partial_\mu(\phi+\delta\phi)\partial^\mu(\phi+\delta\phi)\\
	&= \partial_\mu\phi\partial^\mu\phi+\partial_\mu\phi\partial^\mu\delta\phi+\partial_\mu\delta\phi\partial^\mu\phi+\partial_\mu\delta\phi\partial^\mu\delta\phi\\
	&= \partial_\mu\phi\partial^\mu\phi+2\partial_\mu\phi\partial^\mu\delta\phi
\end{align}

and

\begin{align}
	\frac{\Lambda^4}{2f^2}\phi^2 \rightarrow \frac{\Lambda^4}{\partial f^2}\phi^2 + \frac{\Lambda^4}{2f^2}\delta\phi\cdot\phi+\frac{\Lambda^4}{2f^2}\phi\cdot\delta\phi
\end{align}

and

\begin{align}
	S[\phi+\delta\phi]-S[\phi] &= \int [\partial_\mu\phi\partial^\mu\delta\phi-\frac{\Lambda^4}{f^2}\delta\phi\cdot\phi]\dd^4x\\
	&= \int [-\partial_\mu\partial^\mu\phi-\frac{\Lambda^4}{f^2}\phi]\delta\phi\dd^4x
\end{align}

having used integration by parts. Therefore

\begin{align}
	0=\diffp{S}{\phi}\rightarrow\partial_\mu\partial^\mu\phi+\frac{\Lambda^4}{f^2}\phi=0
\end{align}

\emph{make sure you know how to integrate by parts like this for the exam.}

\subsubsection{Reflections}

The Lagrangian in the next order is

\begin{equation}
  \lagr \approx \frac{1}{2}\partial_\mu\phi\partial^\mu\phi-\frac{1}{2}\frac{\Lambda^4}{f^2}\phi^2+\frac{1}{4!}\frac{\Lambda^4}{f^4}\phi^4
\end{equation}

and we can see it is symmetric for reflections because of the powers of everything.

\subsubsection{another order}

The Lagrangian is

\begin{equation}
  \lagr \approx \frac{1}{2}\partial_\mu\phi\partial^\mu\phi-\frac{1}{2}\frac{\Lambda^4}{f^2}\phi^2+\frac{1}{4!}\frac{\Lambda^4}{f^4}\phi^4 - \frac{1}{6!}\frac{\Lambda^4}{f^6}\phi^6
\end{equation}

No, instead of the elements containing the Lambdas we could have placed generic constants.

\subsubsection{Different Lagrangian}

\begin{equation}
  \lagr \approx \frac{1}{2}\partial_\mu\phi_1\partial^\mu\phi_1+\frac{1}{2}\partial_\mu\phi_2\partial^\mu\phi_2 - \frac{\Lambda^4}{f^2}\frac{1}{2}(\phi_1^2+\phi_2^2)-\frac{\Lambda^4}{f^2}\phi_1\phi_2
\end{equation}

We can see it is not symmetric for flips. Using E-L to find the equations of motion

\begin{equation}
  \diffp{\lagr}{\phi_i}=\partial_\mu\ins{\diffp{\lagr}{\partial_\mu\phi_i}}
\end{equation}

and inserting out specific Lagrangian

\begin{align}
	-\ins{\ins{\frac{\Lambda^4}{f^2}}(\phi_1+\phi_2)} &= \partial^\mu\partial_\mu\phi_1\\
	-\ins{\ins{\frac{\Lambda^4}{f^2}}(\phi_1+\phi_2)} &= \partial^\mu\partial_\mu\phi_2
\end{align}

\subsection{Radiation example}

Two electric dipoles are wobbling with frequency $\omega$ but they are $\pi/2$ off phase with each other. Their amplitude is identical and equal to $P_0$ but their dipole moment has angle $\psi$ between them.

\begin{enumerate}
	\item Find $\diff{\bar P}{\Omega}$.
	\item Find the total power $\bar P=\frac{1}{T}\int P(t)\dd t$.
\end{enumerate}

\subsubsection{part 1}

The first dipole moment can be chosen to be

\begin{equation}
  P_1=P_0\cos(\omega t)\hat x
\end{equation}

and since there is a phase difference of an exact amount we then select

\begin{equation}
  P_2 = P_0\sin(\omega t)[\cos\psi\hat x + \sin\psi\hat y]
\end{equation}

and the total dipole moment is

\begin{equation}
  P_{tot} = P_1+P_2
\end{equation}

The total power output is given by

\begin{equation}
  P = \int\dd s S
\end{equation}

where

\begin{equation}
  B = \frac{1}{4\pi cr}\ddot d(t_{ret})\times\hat r
\end{equation}

and

\begin{equation}
	S=\frac{1}{4\pi r^2c^3}\left|\hat r\times\ddot d(t_{ret})\right|^2\hat r
\end{equation}

We find 

\begin{equation}
  \ddot d = -\omega^2 P_0[\cos(\omega t)\hat x+\sin(\omega t)[\hat x\cos\psi+\hat y\sin\psi]]
\end{equation}

Now there's a terrible change of coordinates to spherical which I won't type, you're welcome to change from cartesian to spherical then do the cross product, it's straightforward, just long. Inserting the horrible long expression we find

\begin{align}
  |\ddot d\times \hat r| &= \omega^2P_0[[-\hat \phi\cos\theta\cos\phi(\cos(\omega t)+\sin(\omega t)\cos\psi)\\
  &+ \cos\theta\sin\phi\sin(\omega t)\sin\psi]\\
  &+ \hat\theta[\sin\phi(\cos(\omega t)+\sin(\omega t)\cos\psi)+\cos\phi\sin(\omega t)\sin\psi]]
\end{align}

and

\begin{align}
	|\hat r \times \ddot \phi|^2 &= \omega^4P_0^2(\cos^2\theta\cos^2\phi(\cos^2(\omega t)+\sin^2(\omega t)\cos^2\psi+2\cos(\omega t)\sin(\omega t)\cos\psi)\\
	&+ \cos^2\theta\sin^2\psi\sin^2(\omega t)\sin^2\psi\\
	&+ 2\cos\theta\sin\phi\sin(\omega t)\sin\psi\cos\theta\cos\phi(\cos(\omega t)+\sin(\omega t)\cos\psi)\\
	&+...
\end{align}

Skipping this mess we find

\begin{equation}
  \diff{\bar P}{\Omega}=\frac{P_0^2\omega^4}{32\pi^2c^2}[2-\sin^2\theta(\cos^2\phi+\cos^2(\theta-\phi))]
\end{equation}

\subsection{Electrostatics example}

A spherical shell of radius a is given with the potential V on its surface. Inside the shell there is a wire with a charge density $\lambda$.

\begin{itemize}
	\item Write the Green's function for the problem (inside the sphere).
	\item Find the potential
\end{itemize}

\subsubsection{Green's function}

We want the Dirichlet Green's for a sphere. We'll use the method of images to find it and find

\begin{equation}
  G(r,r') = \frac{1}{|r-r'|}-\frac{a/r'}{|r-\ins{\frac{a}{r'}}^2r'|}
\end{equation}

Alternatively we could have solved this in a different method. We'll define gamma to be the angle between r and r':

\begin{equation}
  \cos\gamma =\hat r\cdot\hat r'
\end{equation}

thus

\begin{equation}
  G(r,r') = \frac{1}{\sqrt{r^2+r'^2-2rr'\cos\gamma}}-\frac{1}{\sqrt{\ins{\frac{rr'}{a}}^2+a^2-2rr'\cos\gamma}}
\end{equation}

\subsubsection{Finding the potential}

\begin{equation}
  \phi(r) = \int\limits_V \dd v' \rho(r')G(r,r') - \frac{1}{4\pi}\int\limits_{\partial V}\dd a' V\diffp{G}{n}
\end{equation}

Marking the first integral $\phi_\rho$ and the second $\phi_V$. Thus we'll first find the charge density and normal vector directed derivative.

\begin{align}
	\diffp{G}{n'}=\left.\diffp{G}{r'}\right|_{r'=a}=\frac{a^2-r^2}{a(r^2+a^2-2ar\cos\gamma)^{3/2}}
\end{align}

and

\begin{equation}
  \rho(x) = \frac{\lambda}{r^2\sin\theta}\delta(\phi)[\delta(\theta)+\delta(\theta-\pi)]
\end{equation}

We'll next notice that

\begin{equation}
  \cos\gamma = \begin{cases}
  	\cos\theta & \theta'=0 \rightarrow z>0\\
  	-\cos\theta & \theta' = \pi \rightarrow z<0
  \end{cases}
\end{equation}

thus

\begin{align}
	\phi_\rho &= \lambda\int\limits_0^a\dd r'\left[\frac{1}{\sqrt{r^2+r'^2-2rr'\cos\theta}}-\frac{1}{\sqrt{\frac{r^2r'^2}{a^2}+a^2-2rr'\cos\theta}}\right.\\
	&+\left.\frac{1}{\sqrt{r^2+r'^2+2rr'\cos\theta}}-\frac{1}{\sqrt{\frac{r^2r'^2}{a^2}}+a^2+2rr'\cos\theta}\right]
\end{align}

\emph{And that's E\&M.}


\vskip7ex
\centering
* * *
%
\end{document}%