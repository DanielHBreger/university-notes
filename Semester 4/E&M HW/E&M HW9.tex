\documentclass[11pt]{penrose}
\usepackage{amsmath}
\usepackage{amsfonts}
\usepackage{amssymb}
\usepackage{tensor}
\usepackage{mathtools}
\usepackage{witharrows}
\usepackage{diffcoeff}
\usepackage{cancel}
\newcommand\numberthis{\addtocounter{equation}{1}\tag{\theequation}}
\newcommand{\tr}[2]{\tensor{#1}{#2}}
\newcommand{\mtr}[1]{\begin{pmatrix}#1\end{pmatrix}}
\newcommand\dd{\mathop{}\!\mathrm{d}}
\newcommand{\ins}[1]{\left(#1\right)}
\newcommand{\lagr}{\mathcal{L}}
\newcommand{\holdpage}[0]{d\\d\\d\\d\\d\\d\\d\\d\\d\\d\\d\\d\\d\\d\\d\\d\\d\\d\\d\\d\\d}

\title[E\&M 2025 HW9]{E\&M 2025 HW9}
\author{Daniel Haim Breger, 316136944}
\affiliation{Technion}
\date{\today}
\begin{document}

\maketitle

\section{Potential with Poisson}
\begin{colbox}
	\begin{itemize}
		\item Develop the function $\frac{1}{|x-x'|}$ to third order around $x'=0$.
		\item Use the superposition solution of the Poisson equation \begin{equation}\label{poissonsolution}
			\phi(\bold x)=\int\dd^3x\frac{\rho(\bold x')}{|x-x'|}
		\end{equation} and the approximation you found to reach the expression  \begin{equation}
			\phi(x)=\frac{q}{|x|} + \frac{\bold x \cdot \bold d}{|\bold x|^3}+Q_{ij}\frac{x^ix^j}{|\bold x|^5}
		\end{equation} where \begin{align}
			q &= \int\dd^3x'\rho(\bold x')\\
			\bold d &= \int\dd^3x'\rho(\bold x')\bold x'\\
			Q_{ij} &= \frac{1}{2}\int\dd^3x'\rho(\bold x')(3x'_ix'_j-r'^2\delta_{ij})
		\end{align}
	\end{itemize}
\end{colbox}

\subsection{Third order development}

We'll mark the function we need to develop as $f(x')$ for convenience.

\begin{align}
	f(x') &= (|\bold x-\bold x'|)^{-1} = \left(\sqrt{(\bold x-\bold x')^2}\right)^{-1} = \left((\bold x-\bold x')^2\right)^{-1/2}\\
	f(0)&=\frac{1}{|\bold x|}\\
	\left.f'(x')\right|_{x'=0} &= \left.-\frac{1}{2}\cdot 2(\bold x-\bold x')\cdot\left((\bold x-\bold x')^2\right)^{-3/2}\right|_{x'=0} = \left.\frac{\bold x-\bold x'}{(\bold x-\bold x')^{3/2}}\right|_{x'=0} = \frac{\bold x}{|\bold x|^3}\\
	\left.f''(x')\right|_{x'=0} &= \diffp{f'(x_i')}{x_j'}=\diffp*{\frac{x_i-x'_i}{(\bold x-\bold x')^{3/2}}}{x'_j}\\
	&= \frac{\diffp*{(x_i-x_i')(\bold x-\bold x')^{3/2}}{x'_j}-(x_i-x_i')\diffp*{(\bold x-\bold x')^{3/2}}{x'_j}}{(\bold x-\bold x')^3}\\
	&= \frac{\delta_{ij}(\bold x-\bold x')^{3/2}-(x_i-x_i')\cdot\frac{3}{2}(x_j-x_j')\cdot 2 (\bold x-\bold x')^{1/2}}{(\bold x-\bold x')^3}\\
	&= \frac{|\bold x-\bold x'|^3\delta_{ij}-3(x_i-x'_i)(x_j-x'_j)|\bold x-\bold x'|}{|\bold x-\bold x'|^6}\\
	&= \left.\frac{\delta_{ij}}{|\bold x-\bold x'|^3}-\frac{3(x_i-x'_i)(x_j-x'_j)}{|\bold x-\bold x'|^5}\right|_{x'=0}\\
	&= \frac{\delta_{ij}}{|\bold x|^3}-\frac{3x_ix_j}{|\bold x|^5}
\end{align}

Therefore

\begin{align}\label{taylor}
	\frac{1}{|\bold x-\bold x'|} \approx \frac{1}{|\bold x|}+\frac{x_ix'_i}{|\bold x|^3}+\frac{x_ix_j(3x'_ix'_j-\delta_{ij}|\bold x|^2)}{2|\bold x|^5}
\end{align}

\subsection{Reaching an expression}

We'll insert equation \ref{taylor} into equation \ref{poissonsolution}:

\begin{align}
	\phi(\bold x)&=\int\dd^3x\frac{\rho(\bold x')}{|x-x'|}\\
	&= \int\dd^3x'\rho(\bold x')\cdot\left(\frac{1}{|\bold x|}+\frac{x_ix'_i}{|\bold x|^3}+\frac{x'_ix'_j(3x_ix_j-\delta_{ij}|\bold x|^2)}{2|\bold x|^5}\right)\\
	&= \int\dd^3x'\left(\frac{\rho(\bold x')}{|\bold x|}+\frac{\rho(\bold x')x_ix'_i}{|\bold x|^3}+\frac{\rho(\bold x')x'_ix'_j(3x_ix_j-\delta_{ij}|\bold x|^2)}{2|\bold x|^5}\right)\\
	&= \int\dd^3x'\frac{\rho(\bold x')}{|\bold x|}+\int\dd^3x'\frac{\rho(\bold x')x_ix'_i}{|\bold x|^3}+\int\dd^3x'\frac{\rho(\bold x')x'_ix'_j(3x_ix_j-\delta_{ij}|\bold x|^2)}{2|\bold x|^5}\\
	&= \frac{q}{|\bold x|}+\int\dd^3x'\frac{\rho(\bold x')\bold x\cdot \bold x'}{|\bold x|^3}+\int\dd^3x'\frac{\rho(\bold x')x'_ix'_j(3x_ix_j-\delta_{ij}|\bold x|^2)}{2|\bold x|^5}\\
	&= \frac{1}{|\bold x|}+\frac{\bold x\cdot \bold d}{|\bold x|^3}+Q_{ij}\frac{x_ix_j}{|\bold x|^5}
\end{align}

\clearpage
\section{Fun with multipoles}
\begin{colbox}
	Calculate the potential for a distant field in leading order for the three configurations.
\end{colbox}

First we'll find the volumetric charge distribution, since all three cases look similar we assume can find a general one. All the charges are the same distance from the origin and on the same plane and alternate in sign, and are equally spaced along the $\varphi$ axis. Therefore

\begin{equation}
	\rho(r,\theta,\varphi)=\frac{\delta(r-R)\delta(\theta-\frac{\pi}{2})}{R^2}\sum\limits_{n=0}^{N-1}(-1)^n\delta\left(\varphi-\frac{2\pi n}{N}\right)
\end{equation}

We divided by $R^2$ because as we remember from previous homework the 3d delta conversion from cartesian to spherical needs a correction, and the last element being a sum for all the charges, where N is either 2, 4, or 6 depending on the case.

As we know we can develop the potential using spherical harmonics:

\begin{equation}\label{density2}
  \phi(\bold x)=\sum\limits_{l=0}^\infty\sum\limits_{m=-l}^l\frac{4\pi q_{l,m}}{(2l+1)r^{l+1}}Y_{l,m}(\theta,\varphi)
\end{equation}

where

\begin{equation}
  q_{l,m}=\iiint\limits_{\text{everywhere}}\dd^3\bold xY^*_{l,m}(\bold x)|\bold x|^l\rho(\bold x)
\end{equation}

we'll solve this in spherical coordinates and using the density from equation \ref{density2}:

\begin{align}
	q_{l,m}&=\int\limits_0^{2\pi}\dd\varphi\int\limits_0^\pi\dd\theta\sin\theta\int\limits_0^\infty\dd r r^2Y^*_{l,m}(\theta,\varphi)r^l\rho(r,\theta,\varphi)\\
	&= \int\limits_0^{2\pi}\dd\varphi\int\limits_0^\pi\dd\theta\sin\theta\int\limits_0^\infty\dd r r^{l+2}Y^*_{l,m}(\theta,\varphi)\left(\frac{\delta(r-R)\delta(\theta-\frac{\pi}{2})}{R^2}\sum\limits_{n=0}^{N-1}(-1)^n\delta\left(\varphi-\frac{2\pi n}{N}\right)\right)\\
	&= \frac{1}{R^2}\sum\limits_{n=0}^{N-1}(-1)^n\int\limits_0^{2\pi}\dd\varphi\delta(\varphi-\frac{2\pi n}{N})\int\limits_0^\pi\dd\theta\sin\theta\delta(\theta-\frac{\pi}{2})Y^*_{l,m}(\theta,\varphi)\int\limits_0^\infty\dd rr^{l+2}\delta(r-R)\\
	&= R^l\sum\limits_{n=0}^{N-1}(-1)^nY^*_{l,m}(\theta=\frac{\pi}{2},\varphi=\frac{2\pi n}{N})
\end{align}

Finally before moving to each case we notice that the total charge in all cases is 0, therefore $Y_{0,0}=const.$ in all of these cases so we ignore it.

\subsection{2 charges}

In this case we have $N=2$ therefore

\begin{equation}
  q_{l,m}=R^l\sum\limits_{n=0}^{1}(-1)^nY^*_{l,m}(\theta=\frac{\pi}{2},\varphi=\pi n)
\end{equation}

and

\begin{align}
	Y_{1,-1}=\frac{1}{2}\sqrt{\frac{3}{2\pi}} \sin\theta e^{-i\varphi} && Y_{1,0}=0 && Y_{1,1}=-\frac{1}{2}\sqrt{\frac{3}{2\pi}} \sin\theta e^{i\varphi}
\end{align}

therefore

\begin{equation}
  q_{1,\pm1}=2R\cdot\mp\frac{1}{2}\sqrt{\frac{3}{2\pi}}e^{\pm i\varphi}=\mp R\sqrt{\frac{3}{2\pi}}
\end{equation}

and thus the potential is

\begin{align}
	\phi(r,\theta,\varphi) &= \sum\limits_{m=-1}^1\frac{4\pi q_{1,m}}{3r^2}Y_{1,m}\\
	&= \frac{4\pi R}{3r^2}\sqrt{\frac{3}{2\pi}}\cdot\frac{1}{2}\sqrt{\frac{3}{2\pi}}\sin\theta\left(e^{i\varphi}+e^{-i\varphi}\right)\\
	&= \frac{2R}{r^2}\sin\theta\cos\varphi
\end{align}

\subsection{4 charges}

In this case we have $N=4$ therefore

\begin{equation}
  q_{l,m}=R^l\sum\limits_{n=0}^{3}(-1)^nY^*_{l,m}(\theta=\frac{\pi}{2},\varphi=\frac{\pi n}{2})
\end{equation}

From the symmetry of the problem, specifically that the charges are distributed like a plus shape, and general familiarity with the spherical harmonics, we can tell that $q_{1,l}=0$ because they are anti-symmetric when observing their behavior in rotations of $\frac{\pi}{2}$. This can be easily seen in any of the online animations of spherical harmonics. When 1 quarter becomes positive, the adjacent ones are negative. In conclusion, we begin our search at $l=2$:

\begin{align}
	Y_{2,-2}= \frac{1}{4}\sqrt{\frac{15}{2\pi}}\sin^2\theta e^{-2i\varphi}&& && Y_{2,-1}=\frac{1}{2}\sqrt{\frac{15}{2\pi}}\sin\theta\cos\theta e^{-i\varphi} \\ 
	&&Y_{2,0}=\frac{1}{4}\sqrt{\frac{5}{\pi}}(3\cos^2\theta-1)\\
	Y_{2,1} = -\frac{1}{2}\sqrt{\frac{15}{2\pi}}\sin\theta\cos\theta e^{i\varphi} && && Y_{2,2} = \frac{1}{4}\sqrt{\frac{15}{2\pi}}\sin^2\theta e^{2i\varphi}
\end{align}

therefore

\begin{align}
	q_{2,0} &= R^2\sum\limits_{n=0}^{3}(-1)^nY^*_{2,0}(\theta=\frac{\pi}{2},\varphi=\frac{\pi n}{2})\\
	&= R^2\left[Y^*_{2,0}(\pi/2,0)-Y^*_{2,0}(\pi/2,\pi/2)+Y^*_{2,0}(\pi/2,\pi)-Y^*_{2,0}(\pi/2,3\pi/2)\right]
\end{align}

but we notice that $Y_{2,0}(\theta,\varphi+\pi)=-Y_{2,0}(\theta,\varphi)$ therefore

\begin{equation}
  q_{2,0} = 0
\end{equation}

next, for $q_{2,\pm 1}$, we notice the exact same, therefore 

\begin{equation}
  q_{2,\pm 1}=0
\end{equation}

Finally

\begin{align}
	q_{2,2} &= R^2\sum\limits_{n=0}^{3}(-1)^nY^*_{2,2}(\theta=\frac{\pi}{2},\varphi=\frac{\pi n}{2})\\
	&= R^2\left[Y^*_{2,2}(\frac{\pi}{2},0)-Y^*_{2,2}(\frac{\pi}{2},\frac{\pi}{2})+Y^*_{2,2}(\frac{\pi}{2},\pi)-Y^*_{2,2}(\frac{\pi}{2},\frac{3\pi}{2})\right]\\
	&= R^2\frac{1}{4}\sqrt{\frac{15}{2\pi}}\left[\sin^2\frac{\pi}{2}-\sin^2\frac{\pi}{2} e^{-i\pi}+\sin^2\frac{\pi}{2} e^{-2i\pi}-\sin^2\frac{\pi}{2} e^{-3i\pi}\right]\\
	&= \frac{R^2}{4}\sqrt{\frac{15}{2\pi}}\left[1+1+1+1\right]\\
	&= R^2\sqrt{\frac{15}{2\pi}}
\end{align}

and from the rules of the multipoles we know that $q_{2,-2}=-R^2\sqrt{\frac{15}{2\pi}}$ thus

\begin{equation}
  \phi(x)=\frac{4\pi}{5r^3}R^2\sqrt{\frac{15}{2\pi}}(Y_{2,2}+Y_{2,-2})=\frac{3R^2}{5r^3}\sin^2\theta\cos(2\varphi)
\end{equation}

\subsection{6 charges}

Using the same logic as N=4, we find that all moments with $l=0,1,2$ are all zero, as well as $q_{3,m}$ for all $m\in\{0,1,-1,2,-2\}$. Therefore

\begin{equation}
  q_{3,3}=6R^3\left(\frac{1}{8}\sqrt{\frac{35}{\pi}}\right)=\frac{3R^3}{4}\sqrt{\frac{35}{\pi}}
\end{equation}

Therefore

\begin{align}
  \phi(x)&=\frac{4\pi}{7r^4}\frac{3R^3}{4}\sqrt{\frac{35}{\pi}}(-Y_{3,3}+Y_{3,-3})\\
  &= \frac{3\pi R^3}{7r^4}\sqrt{\frac{35}{\pi}}\cdot\frac{1}{8}\sqrt{\frac{35}{\pi}}\sin^3\theta\cos(3\varphi)\\
  &= \frac{15R^3}{4r^4}\sin^3\theta\cos(3\varphi)
\end{align}


\clearpage
\section{Quadropole}
\begin{colbox}
	Calculate the quadropole moment of the configuration \begin{align*}
		q_1=q && \bold r_1=(1,0,0)\\
		q_2=-q && \bold r_2=(1,1,0)
	\end{align*}
\end{colbox}

First we convert the distribution to a charge density

\begin{equation}
  \rho(\bold x)=q\cdot\delta(z)\delta(x-1)\left(\delta(y)-\delta(y-1)\right)
\end{equation}

and inserting this into the formula of a quadropole:

\begin{align}
	Q_{i,j} &= \frac{1}{2}\int\limits_V\dd^3x\rho(\bold x)(3x_ix_j-r\delta_{ij})\\
	&= \frac{1}{2}\int\limits_V\dd^3x q\cdot\delta(z)\delta(x-1)\left(\delta(y)-\delta(y-1)\right)(3x_ix_j-r\delta_{ij})\\
	&= \left.\frac{q}{2}(3x_ix_j-\delta_{ij})\right|_{(x_1,x_2,x_3)=(1,0,0)}-\left.q(3x_ix_j-2\delta_{ij})\right|_{(x_1,x_2,x_3)=(1,1,0)}\\
	&= \frac{q}{2}\mtr{3-1-(3-2) & 0-3 & 0\\0-3 & -1-(3-2) & 0\\0 & 0 & -1-(-2)}\\
	&= \frac{q}{2}\mtr{1&-3&0\\-3&-2&0\\0&0&1}
\end{align}

\end{document}