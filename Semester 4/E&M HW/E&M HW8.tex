\documentclass[11pt]{penrose}
\usepackage{amsmath}
\usepackage{amsfonts}
\usepackage{amssymb}
\usepackage{tensor}
\usepackage{mathtools}
\usepackage{witharrows}
\usepackage{diffcoeff}
\usepackage{cancel}
\newcommand\numberthis{\addtocounter{equation}{1}\tag{\theequation}}
\newcommand{\tr}[2]{\tensor{#1}{#2}}
\newcommand{\mtr}[1]{\begin{pmatrix}#1\end{pmatrix}}
\newcommand\dd{\mathop{}\!\mathrm{d}}
\newcommand{\ins}[1]{\left(#1\right)}
\newcommand{\lagr}{\mathcal{L}}
\newcommand{\holdpage}[0]{d\\d\\d\\d\\d\\d\\d\\d\\d\\d\\d\\d\\d\\d\\d\\d\\d\\d\\d\\d\\d}

\title[E\&M 2025 HW8]{E\&M 2025 HW8}
\author{Daniel Haim Breger, 316136944}
\affiliation{Technion}
\date{\today}
\begin{document}

\maketitle

\emph{This worksheet was solved mostly in my bomb shelter, if there are mistakes, I blame the neighbor's kids who could not stop screaming. It was 80dB on average.}

\section{Green's function}
\begin{colbox}
	An infinite plane is placed at $z=0$ and held at potential V.
	\begin{itemize}
		\item Write the problem's Green function
		\item Above the plane there is a finite wire charged with charge density $\lambda$. It is parallel to the XY plane above the X axis so that its edges are at $(\pm L/2,0,d)$. Find the potential above the plane. You may leave unsolved integrals.
	\end{itemize}
\end{colbox}

\subsection{Green Function}

Since Green functions only depend on the geometry of the problem, we just need a Green function for an infinite plane. We already solved this in the lecture, thus

\begin{equation}
  G_D=\frac{1}{\sqrt{(x-x')^2+(y-y')^2+(z-z')^2}}-\frac{1}{\sqrt{(x-x')^2+(y-y')^2+(z+z')^2}}
\end{equation}

\subsection{Finite wire}

We'll convert the line charge density to volume charge density

\begin{equation}
  \rho(x,y,z)=\lambda \delta(y)\delta(z-d)\Theta(L/2-x)\Theta(x+L/2)
\end{equation}

The potential is given by

\begin{equation}
  \phi=\iiint\limits_V\dd V'\rho G-\frac{1}{4\pi}\iint\limits_{\partial V}\dd S'\cdot\phi\nabla G
\end{equation}

where V is the top half of space (z>0) and thus the surface is the XY plane. Therefore

\begin{align}
	\iiint\limits_V\dd V'\rho G &= \int\limits_{-\infty}^\infty\dd x' \int\limits_{-\infty}^\infty\dd y' \int\limits_0^\infty \dd z' \lambda \delta(y)\delta(z-d)\Theta(L/2-x)\Theta(x+L/2)G\\
	&= \lambda \int\limits_{-L/2}^{L/2}\dd x' G(x,y,z,x',0,d)\\
	&= \lambda \int\limits_{-L/2}^{L/2}\dd x' \frac{1}{\sqrt{(x-x')^2+y^2+(z-d)^2}}-\frac{1}{\sqrt{(x-x')^2+y'^2+(z+d)^2}}
\end{align}

Which I won't solve.

As for the boundary,

\begin{align}
	\iint\limits_{\partial V}\dd S'\cdot\phi\nabla G &= -V\int\limits_{-\infty}^\infty\dd x'\int\limits_{-\infty}^\infty\dd y'\left.\diffp{G}{z'}\right|_{z'=0}\\
	&= -V \int\limits_{-\infty}^\infty\dd x'\int\limits_{-\infty}^\infty\dd y' \frac{2z}{\left((x-x')^2+(y-y')^2+z^2\right)^{3/2}}
\end{align}

therefore the solution is

\begin{align*}
	\phi &= \lambda \int\limits_{-L/2}^{L/2}\dd x' \frac{1}{\sqrt{(x-x')^2+y^2+(z-d)^2}}-\frac{1}{\sqrt{(x-x')^2+y'^2+(z+d)^2}}\\
	&+ \frac{Vz}{2\pi}\int\limits_{-\infty}^\infty\dd x'\int\limits_{-\infty}^\infty\dd y' \frac{1}{\left((x-x')^2+(y-y')^2+z^2\right)^{3/2}}
\end{align*}

\clearpage
\section{Cartesian separation of variables}
\begin{colbox}
	Two parallel grounded planes are given with separation d. Between the two plates there is a plate charged with surface charge density $\sigma$ which is placed perpendicular to the other plates.
	\begin{itemize}
		\item What are the boundary conditions the potential needs to fulfill between the grounded plates?
		\item Using separation of variables find the potential between the plates.
		\item Find the charge density on the top and bottom plates.
	\end{itemize}
\end{colbox}

\subsection{Boundary conditions}

First it's clear the following must be true:

\begin{align}
	\phi(x,0)=\phi(x,d)=0
\end{align}

Second, we expect the potential to approach 0 at infinity thus

\begin{equation}
  \lim\limits_{x\rightarrow\pm\infty}\phi(x,y)=0
\end{equation}

for all y between the plates. Third, We know there is a difference in the electric field across charged plates, which is the gradient of the potential, thus we also expect to see

\begin{equation}
  -\left.\diffp{\phi}{x}\right|_{x=0^+}+\left.\diffp{\phi}{x}\right|_{x=0^-}=4\pi\sigma
\end{equation}

and finally the potential must be continuous

\begin{equation}
  \phi(0^+,y)=\phi(0^-,y)
\end{equation}

\subsection{Finding the Potential}

From the previous subsection we have a set of equations we need to solve. We'll start by setting the Ansatz

\begin{equation}
  X''Y+XY''=0
\end{equation}

with the following boundary conditions

\begin{align}
	Y(0)=0 && Y(d)=0
\end{align}

which gives us the following, using the known solutions for Sturm-Liouville problems

\begin{equation}
  Y=A\sin(\sqrt{\lambda}y)
\end{equation}

and since Y(d) is zero, but we don't want the trivial solution, we find

\begin{equation}
  \sqrt{\lambda} = \frac{\pi n}{d}
\end{equation}

Therefore

\begin{equation}
  Y_n(y)=\sin(\frac{\pi n}{d}y)
\end{equation}

And thus 

\begin{equation}
  \phi(x,y) = \sum\limits_{n=1}^\infty X(x)Y_n(y)=\sum\limits_{n=1}^\infty X(x)\sin(\frac{\pi n}{d} y)
\end{equation}

but we also know that

\begin{equation}
  \diffp[2]{\phi}{x}+\diffp[2]{\phi}{y}=-4\pi\sigma\delta(x)
\end{equation}

therefore

\begin{equation}
  \sum\limits_{n=1}^\infty (X_n''-\lambda_nX_n)\sin(\frac{\pi n}{d}y)=-4\pi\sigma\delta(x)
\end{equation}

We can treat the entire X part as coefficients of an infinite sum of functions we'll develop using the inner product as in a fourier series:

\begin{align}
	<-4\pi\sigma\delta(x),Y_n(y)> &= -4\pi\sigma\delta(x)\int\limits_0^d\dd y\sin(\frac{\pi n}{d}y)\\
	&= 4\frac{\sigma d}{n}\delta(x)\left.\cos(\frac{\pi n}{d}y)\right|_0^d\\
	&= 4\frac{\sigma d}{n}\delta(x)((-1)^n-1)\\
	&= \begin{cases}
		-8\frac{\sigma d}{n}\delta(x) & n=2k+1\\
		0 & n=2k
	\end{cases}
\end{align}

and we find the normalization required

\begin{align}
	<Y_n(y),Y_n(y)> &= \int\limits_0^d\dd y\sin^2(\frac{\pi n}{d}y)=\frac{d}{2}
\end{align}

therefore

\begin{equation}
  -4\pi\sigma\delta(x)=-\sum\limits_{k=1}^\infty\frac{16\sigma}{2k+1}\delta(x)\sin(\frac{\pi n}{d}y)
\end{equation}

therefore

\begin{equation}
  \sum\limits_{n=1}^\infty (X_n''-\lambda_nX_n) = -\sum\limits_{k=1}^\infty\frac{16\sigma}{2k+1}\delta(x)
\end{equation}

Whenever n is even this simplifies to

\begin{equation}
  X_n''-\lambda_nX_n = 0
\end{equation}

Whose solutions are

\begin{equation}
  X_n(x) = A_ne^{\sqrt{\lambda}x}+Be^{-\sqrt{\lambda}x}
\end{equation}

but from our boundary condition that demands the solution goes to 0 an both infinities, both A and B must be 0, making X also 0. For odd n we get

\begin{equation}\label{diffeq1}
  X_n''-\lambda_nX_n = \frac{16\sigma}{n}\delta(x)
\end{equation}

Which is another ODE whose solution is, outside x=0:

\begin{equation}
  X_n(x)=A_ne^{\sqrt{\lambda}x}+B_ne^{-\sqrt{\lambda}x}
\end{equation}

but since the solution could be different at positive and negative x values, we also have

\begin{equation}
  X_n(x) = C_ne^{\sqrt{\lambda}x}+D_ne^{-\sqrt{\lambda}x}
\end{equation}

for negative x.

Since the solution goes to 0 at both infinities we find $A_n=D_n=0$. thus

\begin{equation}
  X_n(x)=\begin{cases}
  	B_ne^{-\sqrt{\lambda}x} & x>0\\
  	C_ne^{\sqrt{\lambda}x} & x<0
  \end{cases}
\end{equation}

From the demand the solution be continuous we find that $B_n=C_n$ thus

\begin{equation}
  X_n(x)=A_ne^{-\sqrt{\lambda}|x|}
\end{equation}

inserting this back into equation \ref{diffeq1} and integrating over x we get

\begin{equation}
  X_n'(x)-\lambda_n A_n\int\limits_{-\infty}^\infty \dd x e^{-\sqrt{\lambda}|x|} = -\frac{16\sigma}{n}
\end{equation}

therefore

\begin{equation}
  2\sqrt{\lambda_n}A_n = \frac{16\sigma}{n} \rightarrow A_n=\frac{8\sigma d}{\pi n^2}
\end{equation}

so the solution for X for odd n is

\begin{equation}
  X_n(x)=\frac{8\sigma d}{\pi n^2}e^{-\frac{\pi n |x|}{d}}
\end{equation}

and therefore the solution for the potential is

\begin{align}
	\phi(x,y) &= \sum\limits_{n=1}^\infty \frac{8\sigma d}{\pi (2n-1)^2}e^{-\frac{\pi (2n-1) |x|}{d}}\sin(\frac{\pi (2n-1)}{d}y)
\end{align}

\subsection{Charge Density}

Recalling the property of the electric field that it "jumps" at conductors, we'll find the electric field perpendicular to the plates from the potential and continue from there.

\begin{align}
	E_\perp &=-\diffp{\phi}{y}\\
	&= \sum\limits_{n=1}^\infty \frac{8\sigma d}{\pi (2n-1)^2}e^{-\frac{\pi (2n-1) |x|}{d}}\cos(\frac{\pi (2n-1)}{d}y)\cdot\frac{\pi(2n-1)}{d}\\
	&= -8\sigma\sum\limits_{n=1}^\infty\frac{1}{2n-1}e^{-\frac{\pi (2n-1) |x|}{d}}\cos(\frac{\pi (2n-1)}{d}y)
\end{align}

for the bottom plate y=0 since both approaches to 0 give the same:

\begin{equation}
  E_{\perp\downarrow}=-8\sigma\sum\limits_{n=1}^\infty\frac{1}{2n-1}e^{-\frac{\pi (2n-1) |x|}{d}}
\end{equation}

therefore, since the field outside the area we're interested in is 0:

\begin{align}
  \sigma &= \frac{1}{4\pi}\cdot-8\sigma\sum\limits_{n=1}^\infty\frac{1}{2n-1}e^{-\frac{\pi (2n-1) |x|}{d}}\\
  &=-\frac{2\sigma}{\pi}\sum\limits_{n=1}^\infty\frac{1}{2n-1}e^{-\frac{\pi (2n-1) |x|}{d}}
\end{align}

and for the top plate we can conclude from the symmetry of the problem that it must be identical:

\begin{equation}
  \sigma = -\frac{2\sigma}{\pi}\sum\limits_{n=1}^\infty\frac{1}{2n-1}e^{-\frac{\pi (2n-1) |x|}{d}}
\end{equation}

\clearpage
\section{Green function between two spherical shells}
\begin{colbox}
	In class you saw the Dirichlet Green function for a spherical shell with radius $R_0$:
	\begin{equation}
		G_D(x,x')=\sum\limits_{l=0}^\infty\sum\limits_{m=-l}^l\frac{4\pi}{1+2l}Y^*_{l,m}(\theta',\varphi')Y_{l,m}(\theta,\varphi)\times\begin{cases}
			\frac{1}{r}\left(\frac{R_0}{r}\right)^l\left[\left(\frac{r'}{R_0}\right)^l-\left(\frac{R_0}{r'}\right)^{l+1}\right] & r>r'\\
			\frac{1}{r'}\left(\frac{R_0}{r'}\right)^l\left[\left(\frac{r}{R_0}\right)^l-\left(\frac{R_0}{r}\right)^{l+1}\right] & r<r'
		\end{cases}
	\end{equation}
	In this exercise we will find the Dirichlet Green function between two concentric shells with radii a and b.
	\begin{itemize}
		\item Write the Green's function as a sum of spherical harmonics in the following way:
		\begin{equation}
			G_D(x,x')=\sum\limits_{l=0}^\infty\sum\limits_{m=-l}^la_{lm}(r|r',\theta',\varphi')Y_{l,m}(\theta,\varphi)
		\end{equation} Explain why you can write the coefficients using this separation of variables \begin{equation}
			a_{lm}(r|r',\theta',\varphi')=g_l(r,r')Y^*_{lm}(\theta',\varphi')
		\end{equation} and find a differential equation for $g_l$.
		\item Show that the solution to this equation is \begin{equation}
			g_l(r,r')=\begin{cases}
				Ar^l+Br^{-(l+1)} & r<r'\\
				A'r^l+B'r^{-(l+1)} & r>r'
			\end{cases}
		\end{equation}
		\item Insert the appropriate boundary conditions for the problem and write $g_l$ using only 2 constants.
		\item We'll define $r_<\equiv \min\{r,r'\}$, $r_>\equiv \max\{r,r'\}$. From the symmetry of $g_l$ we can show that \begin{equation}
			g_l(r,r')=C\left(r_<^l-\frac{a^{2l+1}}{r_<^{l+1}}\right)\left(\frac{1}{r_>^{l+1}}-\frac{r>^l}{b^{2l+1}}\right)
		\end{equation} find C and write the full Green function
		\item Write the Green function at the limit $a\rightarrow 0,b\rightarrow\infty$
	\end{itemize}
\end{colbox}

\subsection{Green as Spherical Harmonics}

First we know that the following is true

\begin{equation}
  \nabla^2G_D(x,x')=-4\pi\delta(\bold x- \bold x')
\end{equation}

if we rewrite the delta function using spherical harmonics

\begin{equation}
  \delta(\bold x-\bold x') = \frac{1}{r^2}\delta(r-r')\sum\limits_{l=0}^\infty\sum\limits_{m=-l}^lY^*_{l,m}(\theta',\varphi')Y_{l,m}(\theta,\varphi)
\end{equation}

combining we find

\begin{equation}\label{nablagreen}
  \nabla^2G_D(x,x')=\frac{-4\pi}{r^2}\delta(r-r')\sum\limits_{l=0}^\infty\sum\limits_{m=-l}^lY^*_{l,m}(\theta',\varphi')Y_{l,m}(\theta,\varphi)
\end{equation}

We also know that spherical harmonics satisfy

\begin{equation}\label{harmonicsidentity}
  \frac{1}{\sin\theta}\diffp*{\left(\sin\theta\diffp{Y_{l,m}}{\theta}\right)}{\theta}+\frac{1}{\sin^2\theta}\diffp[2]{Y_{l,m}}{\varphi}=-l(l+1)Y_{l,m}
\end{equation}

If we look at the given instruction, we can see

\begin{align}
	\nabla^2G_D(x,x') &= \nabla^2\left(\sum\limits_{l=0}^\infty\sum\limits_{m=-l}^la_{lm}(r|r',\theta',\varphi')Y_{l,m}(\theta,\varphi)\right)\\
	&= \sum\limits_{l=0}^\infty\sum\limits_{m=-l}^l \nabla^2\left( a_{lm}(r|r',\theta',\varphi')Y_{l,m}(\theta,\varphi)\right)
\end{align}

and using the laplacian in spherical coordinates

\begin{align}
	&= \sum\limits_{l=0}^\infty\sum\limits_{m=-l}^l\frac{1}{r}\diffp*[2]{(ra_{lm}Y_{l,m})}{r}+\frac{1}{r^2\sin\theta}\diffp*{\left(\sin\theta\diffp*{(a_{lm}Y_{l,m})}{\theta}\right)}{\theta}+\frac{1}{r^2\sin^2\theta}\diffp*[2]{(a_{lm}Y_{l,m})}{\varphi}\\
	&= \sum\limits_{l=0}^\infty\sum\limits_{m=-l}^l\frac{Y_{l,m}}{r}\diffp*[2]{(ra_{lm})}{r}+\frac{a_{lm}}{r^2}\left(\frac{1}{\sin\theta}\diffp*{\left(\sin\theta\diffp{Y_{l,m}}{\theta}\right)}{\theta}+\frac{1}{\sin^2\theta}\diffp[2]{Y_{l,m}}{\varphi}\right)
\end{align}

Using equation \ref{harmonicsidentity}

\begin{align}
	&= \sum\limits_{l=0}^\infty\sum\limits_{m=-l}^l\frac{Y_{l,m}}{r}\diffp*[2]{(ra_{lm})}{r}-\frac{a_{lm}}{r^2}\left(l(l+1)Y_{l,m}\right)\\
	&= \sum\limits_{l=0}^\infty\sum\limits_{m=-l}^l\frac{1}{r}\left(\diffp*[2]{(ra_{lm})}{r}-\frac{l(l+1)}{r}a_{lm}\right)Y_{l,m}
\end{align}

Next we compare this to equation \ref{nablagreen}

\begin{align}
	\frac{-4\pi}{r^2}\delta(r-r')\sum\limits_{l=0}^\infty\sum\limits_{m=-l}^lY^*_{l,m}(\theta',\varphi')Y_{l,m}(\theta,\varphi) &= \sum\limits_{l=0}^\infty\sum\limits_{m=-l}^l\frac{1}{r}\left(\diffp*[2]{(ra_{lm})}{r}-\frac{l(l+1)}{r}a_{lm}\right)Y_{l,m}\\
	\frac{-4\pi}{r}\delta(r-r')Y^*_{l,m}(\theta',\varphi') &= \diffp*[2]{(ra_{lm})}{r}-\frac{l(l+1)}{r}a_{lm}
\end{align}

From the lefthand's variable dependance we expect to be able to write

\begin{equation}
  a_{lm}=g_{lm}(r,r')Y^*_{l,m}(\theta',\varphi')
\end{equation}

therefore

\begin{align}
	\frac{-4\pi}{r}\delta(r-r')Y^*_{l,m}(\theta',\varphi') &= \diffp*[2]{(ra_{lm})}{r}-\frac{l(l+1)}{r}g_{lm}(r,r')Y^*_{l,m}(\theta',\varphi')\\
	\frac{-4\pi}{r}\delta(r-r') &= \diffp*[2]{(rg_{lm})}{r}-\frac{l(l+1)}{r}g_{lm}
\end{align}

but since this equation doesn't depend on m, we can also say g doesn't depend on m, therefore

\begin{equation}\label{glidentity}
  \diffp*[2]{(rg_l(r,r'))}{r}-\frac{l(l+1)}{r}g_l(r,r')=-\frac{4'pi}{r}\delta(r-r')
\end{equation}

\subsection{finding solution}

Since we're looking for solutions outside $r=r'$ then we can solve the following:

\begin{equation}
  \diffp*[2]{(rg_l(r,r'))}{r}-\frac{l(l+1)}{r}g_l(r,r')=0
\end{equation}

We'll define the following helper function

\begin{equation}
  f_l=rg_l
\end{equation}

thus the equation becomes

\begin{equation}
  \diffp*[2]{(f_l(r,r'))}{r}-\frac{l(l+1)}{r^2}f_l(r,r')=0
\end{equation}

rearranging

\begin{equation}
  r^2\diffp*[2]{(f_l(r,r'))}{r}-l(l+1)f_l(r,r')=0
\end{equation}

This is an euler ODE. The characteristic polynomial is $r^2-r-l(l+1)=0$ whose solutions are

\begin{equation}
  r_{1,2}=\frac{1\pm\sqrt{1+4l(l+1)}}{2}=\frac{1\pm(2l+1)}{2}\rightarrow r=\begin{cases}
  	l+1\\- l
  \end{cases}
\end{equation}

therefore the solution to the ODE is

\begin{equation}
  f_l=Ar^{l+1}+Br^{-l}
\end{equation}

therefore

\begin{equation}
  g_l=Ar^l+Br^{-(l+1)}
\end{equation}

We remember that we need different coefficients for both outside the shell and inside therefore

\begin{equation}
  g_l(r,r')=\begin{cases}
  	Ar^l+Br^{-(l+1)} & r<r'\\
  	A'r^l+B'r^{-(l+1)} & r<r'
  \end{cases}
\end{equation}

\subsection{Boundary Conditions}

We'll require that the Green function becomes zero at both shells, i.e

\begin{align}
	g_l(a,r')=g_l(b,r')=0
\end{align}

therefore

\begin{align}
	Aa^l+Ba^{-(l+1)}=0 &\rightarrow B=-a^{2l+1}A\\
	A'b^l+B'b^{-(l+1)}=0&\rightarrow A'=-b^{-(2l+1)}B'
\end{align}

renaming B' to B for convenience, thus,

\begin{equation}
  g_l(r,r')=\begin{cases}
  	A\left(r^l-a^{2l+1}r^{-(l+1)}\right) & r<r'\\
  	B\left(-b^{-(2l+1)}r^l+'r^{-(l+1)}\right) & r<r'
  \end{cases}
\end{equation}

\subsection{Finding C}

We'll integrate equation \ref{glidentity} around r' and take the limit $\epsilon\rightarrow0$:

\begin{align}
	\int\limits_{r'-\epsilon}^{r'+\epsilon}\dd r \diffp*[2]{(rg_l)}{r}-\int\limits_{r'-\epsilon}^{r'+\epsilon}\dd r \frac{l(l+1)}{r}g_l=-\int\limits_{r'-\epsilon}^{r'+\epsilon}\dd r \frac{4\pi}{r}\delta(r-r')
\end{align}

noticing that the middle element is even in a symmetric segment thus

\begin{align}
	\int\limits_{r'-\epsilon}^{r'+\epsilon}\dd r \diffp*[2]{(rg_l)}{r}=-\int\limits_{r'-\epsilon}^{r'+\epsilon}\dd r \frac{4\pi}{r}\delta(r-r')
\end{align}

thus

\begin{align}\label{limits}
	\left[g_l+r\diffp*{g_l}{r}\right]_{r'-\epsilon}^{r'+\epsilon}&=\left.\diffp*{(rg_l)}{r}\right|_{r'-\epsilon}^{r'+\epsilon}\\
	\left[g_l+r\diffp*{g_l}{r}\right]_{r'-\epsilon}^{r'+\epsilon} &= -\frac{4\pi}{r'}
\end{align}

Now we take the limit.

\begin{align}
	\lim\limits_{\epsilon\rightarrow0}\left[g_l+r\diffp*{g_l}{r}\right]_{r'-\epsilon}^{r'+\epsilon}&= \lim\limits_{\epsilon\rightarrow0}\left.\left(g_l+r\diffp*{g_l}{r}\right)\right|_{r'+\epsilon}-\lim\limits_{\epsilon\rightarrow0}\left.\left(g_l+r\diffp*{g_l}{r}\right)\right|_{r'-\epsilon}\\
	&= C\left(r^l-\frac{a^{2l+1}}{r^{l+1}}\right)\left[\left(\frac{1}{r^{l+1}}-\frac{r^l}{b^{2l+1}}\right)+r\left(-(l+1)r^{-(l+2)}-l\frac{r^{l-1}}{b^{2l+1}}\right)\right]\\
	&-C\left(r^{-(l+1)}-\frac{r^l}{b^{2l+1}}\right)\left[\left(r^l-\frac{a^{2l+1}}{r^{l+1}}\right)+r\left(lr^{l-1}-(l+1)\frac{a^{2l+1}}{r^{l+2}}\right)\right]
\end{align}

inserting into equation \ref{limits} we must find that

\begin{equation}
  C=\frac{4\pi}{(2l+1)\left(1-\left(\frac{a}{b}\right)^{2l+1}\right)}
\end{equation}

Therefore the full Green function is

\begin{equation}
  G_D(x,x')=4\pi\sum\limits_{l=0}^\infty\sum\limits_{m=-l}^l\frac{Y^*_{lm}(\theta,\varphi)Y_{lm}(\theta,\varphi)}{(2l+1)\left(1-\left(\frac{a}{b}\right)^{2l+1}\right)}\left(r_<^l-\frac{a^{2l+1}}{r_<^{l+1}}\right)\left(\frac{1}{r_>^{l+1}}-\frac{r_>^l}{b^{2l+1}}\right)
\end{equation}

\subsection{Taking a Limit}

\begin{equation}
 G_D(x,x')= 4\pi\sum\limits_{l=0}^\infty\sum\limits_{m=-l}^l\frac{Y^*_{lm}(\theta,\varphi)Y_{lm}(\theta,\varphi)}{2l+1}\left(\frac{r_<^l}{r_>^{l+1}}\right)
\end{equation}


\clearpage
\section{Alternating potentials on proceeding slices}
\begin{colbox}
	A hollow conductive shell with inner radius a is split into an even number of slices separated by planes coinciding with the z axis, where each plane is separated by an azimuthal angle $\frac{\pi}{n}$ from the next plane. Each slice is held at potential $\pm V$ where the sign alternates between slices.
	\begin{itemize}
		\item Write the potential inside the shell as an infinite sum with separation of variables in spherical coordinates, and write integral expressions for the coefficients of the expression.
		\item Show that the coefficients of $Y_{l,m}$ are zero unless $l+m$ is even.
		\item Show that the problem has symmetry of the form \begin{equation}
			\phi(\varphi)\rightarrow A\phi(\varphi+\Delta\varphi)
		\end{equation} and find A.
		\item Determine for which values of m the coefficient of $Y_{l,m}$ in the expression from the previous part is zero. Write the non-zero coefficients as an integral on $\cos\theta$ only. What is the $Y_{l,m}$ with the smallest l which contributes to the potential?
	\end{itemize}
\end{colbox}

\subsection{Potential in the shell}

The solution for the laplace equation in spherical coordinates is

\begin{equation}
  \phi(r,\theta,\varphi)=\sum\limits_{l=0}^\infty\sum\limits_{m=-l}^l\left(A_{l,m}\left(\frac{r}{a}\right)^l+B_{l,m}\left(\frac{a}{r}\right)^{-(l+1)}\right)Y_{l,m}(\theta,\varphi)
\end{equation}

But since we know the potential is not infinite at the origin, $B_{l,m}=0$. Therefore

\begin{equation}
  \phi(r,\theta,\varphi)=\sum\limits_{l=0}^\infty\sum\limits_{m=-l}^lA_{l,m}\left(\frac{r}{a}\right)^lY_{l,m}(\theta,\varphi)
\end{equation}

For convenience we'll define

\begin{equation}
  \Phi = \begin{cases}
  	V & \frac{2\pi k}{n} < \varphi < \frac{(2k+1)\pi}{n}\\
  	-V & \frac{(2k+1)\pi}{n} < \varphi < \frac{2(k+1)\pi}{n}\\
  \end{cases}
\end{equation}


Next, we know the potential at $r=a$:

\begin{equation}
  \phi(a,\theta,\varphi)=\sum\limits_{l=0}^\infty\sum\limits_{m=-l}^lA_{l,m}Y_{l,m}(\theta,\varphi)= \Phi
\end{equation}

And since the spherical harmonics are a complete orthonormal set, the coefficients thus must be

\begin{equation}
  A_{l,m}=\int\limits_0^{2\pi}\dd \varphi\int\limits_0^\pi \dd\theta\sin\theta Y^*_{l,m}(\theta,\varphi)\Phi
\end{equation}

\subsection{Coefficients are zero}

We know that the following relation holds for Legendre polynomials:

\begin{equation}
  P_{l,m}(-x)=(-1)^{l+m}P_{l,m}(x)
\end{equation}

and we know that

\begin{equation}
  Y_{l,m} \propto P_{l,m}
\end{equation}

Finally we know that the problem is symmetric for rotations of $\pi$ radians. This is important because $-\cos\theta=\cos(\pi-\theta)$

Therefore we find that

\begin{equation}
  Y_{l,m}(\theta,\varphi)=(-1)^{l+m}Y_{l,m}(\pi-\theta,\varphi)
\end{equation}

Which means that we must have $l+m$ to be even.

\subsection{A Symmetry}

From the question's conditions we find that

\begin{equation}
  V(\varphi+\Delta\varphi)=-V(\varphi)
\end{equation}

We also note that the laplacian operator is invariant for rotational translations, therefore we must find that

\begin{equation}
  \phi(\varphi+\Delta\varphi)=-\phi(\varphi)
\end{equation}

i.e we find that $A=-1$.

\subsection{When it is zero}

Since we found the symmetry for the problem, the solution we found earlier must also satisfy it, therefore the coefficients for each $Y_{l,m}$ must satisfy it.

\begin{equation}
  e^{im\varphi}=-e^{im(\varphi+\frac{\pi}{n})}
\end{equation}

therefore

\begin{equation}
  m=(2k-1)n
\end{equation}

I stop here because I simply can't concentrate enough to finish.

\end{document}